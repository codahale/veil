\section{Passphrase-Based Encryption}\label{sec:veil.pbenc}

\texttt{veil.pbenc} implements a passphrase-based authenticated encryption scheme to encrypt private keys at rest.

\subsection{Expanding A Passphrase}\label{subsec:veil.pbenc-expand}

Expanding a passphrase is described in full in Alg.~\ref{alg:veil.pbenc-expand}\@.

\begin{algorithm}[ht]
    \caption{
        Expanding a passphrase into a keyed duplex given a passphrase $P$, a salt $S$, and a cost parameter $T_{cost}$.
    }
    \begin{algorithmic}
        \Procedure{ExpandPassphrase}{$P, S, T_{cost}$}
            \State $P_N \gets$\Call{NormalizeNFC}{$P$}\Comment{Normalize the passphrase.}
            \State \Call{Absorb}{\texttt{veil.pbenc}}\Comment{Initialize an unkeyed duplex.}
            \State \Call{Absorb}{$P_N$}\Comment{Absorb the passphrase, salt, and cost.}
            \State \Call{Absorb}{$S$}
            \State \Call{Absorb}{$T_{cost}$}
            \ForAll{$i \in 0..2^{T_{cost}}$}
                \State \Call{Absorb}{$i$}\Comment{Absorb the counter.}
                \State $d \gets$ \Call{Squeeze}{$64$} $\modl$\Comment{Squeeze a scalar.}
                \State $Q \gets [d]G$\Comment{Multiply the basepoint by the scalar.}
                \State \Call{Absorb}{$Q$}\Comment{Absorb the point.}
            \EndFor
            \State \Call{Cyclist}{\textsc{SqueezeKey}($64$), $\epsilon$, $\epsilon$}\Comment{Convert to a keyed duplex.}
        \EndProcedure
    \end{algorithmic}
    \label{alg:veil.pbenc-expand}
\end{algorithm}

\subsection{Encrypting A Private Key}\label{subsec:veil.pbenc-encrypt}

Encrypting a private key is described in full in Alg.~\ref{alg:veil.pbenc-encrypt}\@.

\begin{algorithm}[ht]
    \caption{
        Encrypting a private key $d$ given a passphrase $P$ and cost parameter $T_{cost}$.
    }
    \begin{algorithmic}
        \Function{EncryptPrivateKey}{$P, T_{cost}, d$}
            \State $S \rgets \allbits{128}$\Comment{Generate a random salt.}
            \State \Call{ExpandPassphrase}{$P, S, T_{cost}$}\Comment{Expand the passphrase and salt.}
            \State $C \gets$\Call{Encrypt}{$d$}\Comment{Encrypt the private key.}
            \State $T \gets$\Call{Squeeze}{$16$}\Comment{Squeeze an authentication tag.}
            \State \textbf{return} $T_{cost} || S || C || T$
        \EndFunction
    \end{algorithmic}
    \label{alg:veil.pbenc-encrypt}
\end{algorithm}

\subsection{Decrypting A Private key}\label{subsec:veil.pbenc-decrypt}

Decrypting a private key is described in full in Alg.~\ref{alg:veil.pbenc-decrypt}\@.

\begin{algorithm}[ht]
    \caption{
        Decrypting a private key given a passphrase $P$ and ciphertext $C$.
    }
    \begin{algorithmic}
        \Function{DecryptPrivateKey}{$P, C=T_{cost} || S || C || T$}
            \State \Call{ExpandPassphrase}{$P, S, T_{cost}$}\Comment{Expand the passphrase and salt.}
            \State $d' \gets$\Call{Decrypt}{$C$}\Comment{Decrypt the ciphertext.}
            \State $T' \gets$\Call{Squeeze}{$16$}\Comment{Squeeze a counterfactual tag.}
            \If{$T' \checkeq T$}\Comment{Authenticate the ciphertext.}
            \State \textbf{return} $d'$
            \Else
                \State \textbf{return} $\bot$
            \EndIf
        \EndFunction
    \end{algorithmic}
    \label{alg:veil.pbenc-decrypt}
\end{algorithm}

\subsection{Constructive Analysis Of \texttt{veil.pbenc}}\label{subsec:veil.pbenc-analysis}

The \textsc{ExpandPassphrase} procedure implements a novel PBKDF2-style strictly sequential function inspired by
Makwa~\cite{rfc2898, pornin2015}\@.
In addition to absorbing a counter value, each step squeezes a scalar from the duplex state, multiplies the base point
by that scalar, and absorbs the resulting point.
Like PBKDF2, this uses an iterative hash as a CPU-intensive function and has a single cost parameters which is easily
understood;
unlike PBKDF2, the use of a duplex instead of an HMAC eliminates the possibility of cacheable
operations~\cite{visconti2015}\@.
Like Makwa, this adds a number-theoretic component, making inversion of this step equivalent to the discrete logarithm
problem;
unlike Makwa, this construction affords no escrow or fast path possibilities.

\textsc{ExpandPassphrase} is not memory-hard, which may afford a well-funded attacker an advantage by using dedicated
hardware, but in this context the benefits of a simple implementation with a single cost parameter outweigh the benefits
of memory-hardness.
Given that \texttt{veil.pbenc} operations are infrequent and user-local (unlike e.g.\ password verification), a defender
can simply increase the cost parameter to make attacks infeasible.

The remainder of \texttt{veil.pbenc} is a standard Cyclist authenticated encryption scheme.
The duplex state depends on the random salt value and the resulting construction is IND-CCA2 secure.
