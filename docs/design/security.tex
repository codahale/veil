\section{Security Model And Notions}\label{sec:security-model-and-notions}

\subsection{Multi-User Confidentiality}\label{subsec:sec-conf}

To evaluate the confidentiality of a scheme, we consider an adversary $\Adversary$ attempting to
attack a sender and receiver~\cite[p. 44]{baek2010}\@. $\Adversary$ creates two equal-length
messages $(m_0, m_1)$, the sender selects one at random and encrypts it, and $\Adversary$ guesses
which of the two has been encrypted without tricking the receiver into decrypting it for them. To
model real-world possibilities, we assume $\Adversary$ has three capabilities:

\begin{enumerate}
    \item $\Adversary$ can create their own key pairs. Veil does not have a centralized certificate
          authority and creating new key pairs is intentionally trivial.
    \item $\Adversary$ can trick the sender into encrypting arbitrary plaintexts with arbitrary
          public keys. This allows us to model real-world flaws such as servers which return
          encrypted error messages with client-provided data~\cite{yu2004}\@.
    \item $\Adversary$ can trick the receiver into decrypting arbitrary ciphertexts from arbitrary
          senders. This allows us to model real-world flaws such as padding
          oracles~\cite{rizzo2010practical}\@.
\end{enumerate}

Given these capabilities, $\Adversary$ can mount an attack in two different settings: the outsider
setting and the insider setting.

\subsubsection{Outsider Confidentiality}\label{subsubsec:sec-conf-outsider}

In the multi-user outsider model, we assume $\Adversary$ knows the public keys of all users but none
of their private keys~\cite[p. 44]{baek2010}\@.

The multi-user outsider model is useful in evaluating the strength of a scheme against adversaries
who have access to some aspect of the sender and receiver's interaction with messages \@(e.g.\ a
padding oracle) but who have not compromised the private keys of either.

\subsubsection{Insider Confidentiality}\label{subsubsec:sec-conf-insider}

In the multi-user insider model, we assume $\Adversary$ knows the sender's private key in addition
to the public keys of both users~\cite[p. 45--46]{baek2010}\@.

The multi-user insider model is useful in evaluating the strength of a scheme against adversaries
who have compromised a user.

\paragraph{Forward Sender Security}

A scheme which provides confidentiality in the multi-user insider setting is called \emph{forward
    sender secure} because an adversary who compromises a sender cannot read messages that sender
has previously encrypted~\cite{canetti2003}\@.

\subsection{Multi-User Authenticity}\label{subsec:sec-auth}

To evaluate the authenticity of a scheme, we consider an adversary $\Adversary$ attempting to attack
a sender and receiver~\cite[p. 47]{baek2010}\@. $\Adversary$ attempts to forge a ciphertext which
the receiver will decrypt but which the sender never encrypted. To model real-world possibilities,
we again assume $\Adversary$ has three capabilities:

\begin{enumerate}
    \item $\Adversary$ can create their own key pairs.
    \item $\Adversary$ can trick the sender into encrypting arbitrary plaintexts with arbitrary
          public keys.
    \item $\Adversary$ can trick the receiver into decrypting arbitrary ciphertexts from arbitrary
          senders.
\end{enumerate}

As with multi-user confidentiality, this can happen in the outsider setting and the insider setting.

\subsubsection{Outsider Authenticity}\label{subsubsec:sec-auth-outsider}

In the multi-user outsider model, we again assume $\Adversary$ knows the public keys of all users
but none of their private keys~\cite[p. 47]{baek2010}\@.

Again, this is useful to evaluate the strength of a scheme in which $\Adversary$ has some insight
into senders and receivers but has not compromised either.

\subsubsection{Insider Authenticity}\label{subsubsec:sec-auth-insider}

In the multi-user insider model, we assume $\Adversary$ knows the receiver's private key in addition
to the public keys of both users~\cite[p. 48]{baek2010}\@.

\paragraph{Key Compromise Impersonation}

A scheme which provides authenticity in the multi-user insider setting effectively resists \emph{key
    compromise impersonation}, in which $\Adversary$, given knowledge of a receiver's private key,
can forge messages to that receiver from arbitrary senders~\cite{strangio2006}\@. The classic
example is authenticated Diffie-Hellman (e.g.\ RFC 9180~\cite{rfc9180, alwen2021}), in which the
static Diffie-Hellman shared secret point $K=[d_S]Q_R$ is used to encrypt a message and its
equivalent $K'=[d_R]Q_S$ is used to decrypt it. An attacker in possession of the receiver's
private key $d_R$ and the sender's public key $Q_S$ can simply encrypt the message using
$K'=[d_R]Q_S$ without ever having knowledge of $d_S$. Digital signatures are a critical element
of schemes which provide insider authenticity, as they give receivers a way to verify the
authenticity of a message using authenticators they \@(or an adversary with their private key)
could never construct themselves.

\subsection{Insider vs. Outsider Security}\label{subsec:security-insider-vs-outsider}

The multi-receiver setting motivates a focus on insider security over the traditional emphasis on
outsider security (contra~\cite[p. 26]{an2010}\cite[p. 46]{baek2010}; see~\cite{badertscher2018}).
Given a probability of an individual key compromise $P$, a multi-user system of $N$ users has an
overall $1-{(1-P)}^N$ probability of at least one key being compromised. A system with an
exponentially increasing likelihood of losing all confidentiality and authenticity properties is not
acceptable.

\subsection{Indistinguishable From Random Noise}\label{subsec:security-indistinguishable}

Indistinguishability from random noise is a critical property for censorship-resistant
communication~\cite{bernstein2013}\@:

\begin{displayquote}
    Censorship-circumvention tools are in an arms race against censors. The censors study all
    traffic passing into and out of their controlled sphere, and try to disable
    censorship-circumvention tools without completely shutting down the Internet. Tools aim to shape
    their traffic patterns to match unblocked programs, so that simple traffic profiling cannot
    identify the tools within a reasonable number of traces; the censors respond by deploying
    firewalls with increasingly sophisticated deep-packet inspection.

    Cryptography hides patterns in user data but does not evade censorship if the censor can
    recognize patterns in the cryptography itself.
\end{displayquote}

\subsection{Limited Deniability}\label{subsec:security-deniability}

The inability of a receiver (or an adversary in possession of a receiver's private key) to prove the
authenticity of a message to a third party is critical for privacy. Other privacy-sensitive
protocols achieve this by forfeiting insider authenticity or authenticity
altogether~\cite{borisov2004}\@. Veil achieves a limited version of deniability: a receiver can only
prove the authenticity of a message to a third party by revealing their own private key. This deters
a dishonest receiver from selectively leaking messages and requires all-or-nothing disclosure from
an adversary who compromises an honest receiver.
