\section{Cryptographic Primitives}\label{sec:cryptographic-primitives}

In the interests of cryptographic minimalism, Veil uses just two distinct cryptographic primitives:

\begin{itemize}
    \item Cyclist/Keccak-\emph{f}[1600,12] for confidentiality, authentication, and
    integrity~\cite{daemen2020,bertoni2018}.
    \item Ristretto255 for key agreement and authenticity~\cite{deValence2020}.
\end{itemize}

\subsection{Cyclist/Keccak-\emph{f}[1600,12]}\label{subsec:cyclist}

Cyclist is a permutation-based cryptographic duplex, a cryptographic primitive that provides symmetric-key
confidentiality, integrity, and authentication via a single object~\cite{daemen2020}.
Duplexes offer a way to replace complex, ad-hoc constructions combining encryption algorithms, cipher modes,
AEADs, MACs, and hash algorithms using a single primitive~\cite{daemen2020, bertoni2011duplex}.
Duplexes have security properties which reduce to the properties of the cryptographic sponge, which themselves reduce to
the strength of the underlying permutation~\cite{bertoni2008}.

Veil uses the Cyclist construction parameterized with $f$=Keccak-\emph{f}[1600,12], $R_\text{hash}$=1344,
$R_\text{kin}$=1568, $R_\text{kout}$=1408, and $\ell_\text{ratchet}$=16 (as calculated per~\cite{bertoni2015keyak} with
$c=192$).
The Keccak-\emph{f}[1600,12] permutation is the basis of the KangarooTwelve hash algorithm~\cite{bertoni2018}.
It has a 1600-bit width, like Keccak-\emph{f}[1600] (the basis of SHA-3), but with a reduced number of rounds for speed.
This targets a 128-bit security level and is $~4x$ faster than SHA-256, $~5x$ faster than ChaCha20Poly1305, and $~7.5x$
faster than AES-128-GCM in software.

Veil's security assumes that Cyclist's \textsc{Encrypt} operation is IND-CPA secure, its
\textsc{Squeeze} operation is sUF-CMA secure, and its
\textsc{Encrypt}/\textsc{Squeeze}-based authenticated encryption construction is IND-CCA2 secure.

\subsection{Ristretto255}\label{subsec:ristretto255}

Ristretto255 uses a safe curve, is a prime-order cyclic group, has non-malleable encodings, and has no
co-factor concerns.
This allows for the use of a wide variety of cryptographic constructions built on group operations.
It targets a 128-bit security level, lends itself to constant-time implementations, and can run in constrained
environments~\cite{deValence2018}.

Veil's security assumes that the Gap Discrete Logarithm and Gap Diffie-Hellman problems are hard relative to
Ristretto255.
