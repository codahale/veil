\section{Cryptographic Primitives}\label{sec:cryptographic-primitives}

In the interests of cryptographic minimalism, Veil uses just two distinct cryptographic primitives:

\begin{itemize}
    \item Keccyak-128~\cite{daemen2020,bertoni2018} for confidentiality, authentication, and
          integrity.
    \item jq255e for key agreement and authenticity~\cite{pornin2022}\@.
\end{itemize}

\subsection{Keccyak-128}\label{subsec:keccyak}

Keccyak-128 is the adaptation of the Cyclist duplex construction from Xoodyak~\cite{daemen2020} to
the Keccak-\emph{p}[1600,12] permutation instead of Xoodoo (thus ``Keccyak'').

Cyclist is a permutation-based cryptographic duplex, a cryptographic primitive that provides
symmetric-key confidentiality, integrity, and authentication via a single
object~\cite{daemen2020}\@. Duplexes offer a way to replace complex, ad-hoc constructions combining
encryption algorithms, cipher modes, AEADs, MACs, and hash algorithms using a single
primitive~\cite{daemen2020, bertoni2011duplex}\@. Duplexes have security properties which reduce to
the properties of the cryptographic sponge, which themselves reduce to the strength of the
underlying permutation~\cite{bertoni2008}\@.

The Keccak-\emph{p}[1600,12] permutation is the basis of the KangarooTwelve hash
algorithm~\cite{bertoni2018} and allows for much higher throughput in software than the Xoodoo
permutation at the expense of requiring a larger state. It has a 1600-bit width, like
Keccak-\emph{f}[1600] (the basis of SHA-3)\@, but with a reduced number of rounds for speed.
Xoodyak's Cyclist parameters of $R_\text{hash}=b-256$, $R_\text{kin}=b-32$, $R_\text{kout}=b-192$,
and $\ell_\text{ratchet}=16$ are adapted for the larger permutation width of $b=1600$. The resulting
construction targets a 128-bit security level and is $\sim$4x faster than SHA-256, $\sim$5x faster
than ChaCha20Poly1305, and $\sim$7.5x faster than AES-128-GCM in software.

Veil's security assumes that Cyclist's \textsc{Encrypt} operation is IND-CPA secure, its
\textsc{Squeeze} operation is sUF-CMA secure, and its \textsc{Encrypt}/\textsc{Squeeze}-based
authenticated encryption construction is IND-CCA2 secure.

\subsection{jq255e}\label{subsec:jq255e}

jq255e is a double-odd elliptic curve selected for efficiency and
simplicity~\cite{pornin2020do,pornin2022}\@. It provides a prime-order group, has non-malleable
encodings, and has no co-factor concerns. This allows for the use of a wide variety of cryptographic
constructions built on group operations. It targets a 128-bit security level, lends itself to
constant-time implementations, and can run in constrained environments.

Veil's security assumes that the Gap Discrete Logarithm and Gap Diffie-Hellman problems are hard
relative to jq255e.
