\section{Cryptographic Primitives}\label{sec:cryptographic-primitives}

In the interests of cryptographic minimalism, Veil uses just two distinct cryptographic primitives:

\begin{itemize}
    \item Xoodyak~\cite{daemen2020} for confidentiality, authentication, and integrity.
    \item Ristretto255~\cite{deValence2020} for key agreement and authenticity.
\end{itemize}

\subsection{Xoodyak}\label{subsec:xoodyak}

Xoodyak is a cryptographic duplex, a cryptographic primitive that provides symmetric-key confidentiality,
integrity, and authentication via a single object.
Duplexes offer a way to replace complex, ad-hoc constructions combining encryption algorithms, cipher modes,
AEADs, MACs, and hash algorithms using a single primitive~\cite{daemen2020, bertoni2011duplex}.

Duplexes have security properties which reduce to the properties of the cryptographic sponge, which themselves reduce to
the strength of the underlying permutation~\cite{bertoni2008}.
Xoodyak is based on the Xoodoo permutation, an adaptation of the
Keccak-\emph{p} permutation (upon which SHA-3 is built) for lower-resource environments.
While Xoodyak is not standardized, it is currently a finalist in the NIST Lightweight Cryptography
standardization process.
It targets a 128-bit security level, lends itself to constant-time implementations, and can run in constrained
environments~\cite{daemen2020}.

Veil's security assumes that Xoodyak's \textsc{Encrypt} operation is IND-CPA secure, its
\textsc{Squeeze} operation is sUF-CMA secure, and its
\textsc{Encrypt}/\textsc{Squeeze}-based AEAD construction is IND-CCA2 secure.

\subsection{Ristretto255}\label{subsec:ristretto255}

Ristretto255 uses a safe curve, is a prime-order cyclic group, has non-malleable encodings, and has no
co-factor concerns.
This allows for the use of a wide variety of cryptographic constructions built on group operations.
Like Xoodyak, it targets a 128-bit security level, lends itself to constant-time implementations, and can
run in constrained environments~\cite{deValence2018}.

Veil's security assumes that the Gap Discrete Logarithm and Gap Diffie-Hellman problems are hard relative to
Ristretto255.
