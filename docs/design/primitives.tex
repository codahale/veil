\section{Cryptographic Primitives}\label{sec:cryptographic-primitives}

In the interests of cryptographic minimalism, Veil uses just two distinct cryptographic primitives:

\begin{itemize}
    \item Cyclist/Keccak-\emph{f}[1600,12]~\cite{daemen2020,bertoni2018} for confidentiality, authentication, and
    integrity.
    \item Ristretto255~\cite{deValence2020} for key agreement and authenticity.
\end{itemize}

\subsection{Cyclist/Keccak-\emph{f}[1600,12]}\label{subsec:cyclist}

Cyclist is a permutation-based cryptographic duplex, a cryptographic primitive that provides symmetric-key
confidentiality, integrity, and authentication via a single object~\cite{daemen2020}.
Duplexes offer a way to replace complex, ad-hoc constructions combining encryption algorithms, cipher modes,
AEADs, MACs, and hash algorithms using a single primitive~\cite{daemen2020, bertoni2011duplex}.
Duplexes have security properties which reduce to the properties of the cryptographic sponge, which themselves reduce to
the strength of the underlying permutation~\cite{bertoni2008}.

The Keccak-\emph{f}[1600,12] permutation is the basis of the KangarooTwelve hash algorithm~\cite{bertoni2018}.
It has a 1600-bit width, like Keccak-\emph{f}[1600] (the basis of SHA-3), but with a reduced number of rounds for speed.
It's used to instantiate Cyclist objects with an unkeyed hash rate of 1088 bits.
In the keyed mode, the squeeze rate is 1536 bits, the absorb rate is 800 bits, and the ratchet rate is 256 bits.
This targets a 256-bit security level and is roughly 40\% faster than ChaCha20Poly1305 in software.

Veil's security assumes that Cyclist's \textsc{Encrypt} operation is IND-CPA secure, its
\textsc{Squeeze} operation is sUF-CMA secure, and its
\textsc{Encrypt}/\textsc{Squeeze}-based AEAD construction is IND-CCA2 secure.

\subsection{Ristretto255}\label{subsec:ristretto255}

Ristretto255 uses a safe curve, is a prime-order cyclic group, has non-malleable encodings, and has no
co-factor concerns.
This allows for the use of a wide variety of cryptographic constructions built on group operations.
It targets a 128-bit security level, lends itself to constant-time implementations, and can run in constrained
environments~\cite{deValence2018}.

Veil's security assumes that the Gap Discrete Logarithm and Gap Diffie-Hellman problems are hard relative to
Ristretto255.
