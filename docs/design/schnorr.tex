\section{Digital Signatures}\label{sec:veil.schnorr}

\texttt{veil.schnorr} implements a Schnorr digital signature scheme.

\subsection{Signing A Message}\label{subsec:veil.schnorr-sign}

Signing a message is described in full in Alg.~\ref{alg:veil.schnorr-sign}.

\begin{algorithm}[!htp]
    \caption{
        Signing a message $M$ with a key pair $(d, Q)$.
    }
    \begin{algorithmic}
        \Function{Sign}{$(d_S, Q_S), M$}
            \State \Call{Absorb}{\texttt{veil.schnorr}}\Comment{Initialize an unkeyed duplex.}
            \State \Call{Absorb}{$Q$}\Comment{Absorb the signer's public key.}
            \State \Call{Absorb}{$M$}\Comment{Absorb the message.}
            \State
            \Clone \Comment{Clone the duplex's state.}
            \State \Call{Absorb}{$d$}\Comment{Absorb the sender's private key.}
            \State $v \rgets \allbits{512}$\Comment{Generate a random value.}
            \State \Call{Absorb}{$v$}\Comment{Absorb the random value.}
            \State $k \gets$ \Call{Squeeze}{$32$} $\modl$
            \State \textbf{yield} $k$\Comment{Yield a hedged commitment scalar.}
            \End
            \State
            \State \Call{Cyclist}{\textsc{SqueezeKey}($64$), $\epsilon$, $\epsilon$}\Comment{Convert to a keyed duplex.}
            \State
            \State $I \gets [k]G$\Comment{Calculate the commitment point.}
            \State $S_0 \gets $ \Call{Encrypt}{$I$}\Comment{Encrypt the commitment point.}
            \State
            \State $r \gets$ \Call{Squeeze}{$32$} $\modl$\Comment{Squeeze a challenge scalar.}
            \State $s \gets dr + k$\Comment{Calculate the proof scalar.}
            \State $S_1 \gets $ \Call{Encrypt}{$s$}\Comment{Encrypt the proof scalar.}
            \State
            \State \textbf{return} $S=S_0 || S_1$
        \EndFunction
    \end{algorithmic}
    \label{alg:veil.schnorr-sign}
\end{algorithm}

\subsection{Verifying A Signature}\label{subsec:veil.schnorr-verify}

Verifying a signature is described in full in Alg.~\ref{alg:veil.schnorr-verify}.

\begin{algorithm}[!htp]
    \caption{
        Verifying a signature $S$ with a message $M$ and a public key $Q$.
    }
    \begin{algorithmic}
        \Function{Verify}{$Q, M, S=S_0||S_1$}
            \State \Call{Absorb}{\texttt{veil.schnorr}}\Comment{Initialize an unkeyed duplex.}
            \State \Call{Absorb}{$Q$}\Comment{Absorb the signer's public key.}
            \State \Call{Absorb}{$M$}\Comment{Absorb the message.}
            \State
            \State \Call{Cyclist}{\textsc{SqueezeKey}($64$), $\epsilon$, $\epsilon$}\Comment{Convert to a keyed duplex.}
            \State
            \State $I \gets $ \Call{Decrypt}{$S_0$}\Comment{Decrypt the commitment point.}
            \State $r \gets$ \Call{Squeeze}{$32$} $\modl$\Comment{Squeeze a challenge scalar.}
            \State
            \State $s \gets $ \Call{Decrypt}{$S_1$}\Comment{Decrypt the proof scalar.}
            \State $I' \gets [s]G - [r]Q$\Comment{Calculate the counterfactual commitment point.}
            \State
            \State \textbf{return} $I' \checkeq I$\Comment{The signature is valid if both points are equal.}
        \EndFunction
    \end{algorithmic}
    \label{alg:veil.schnorr-verify}
\end{algorithm}

\subsection{Constructive Analysis Of \texttt{veil.schnorr}}\label{subsec:veil.schnorr-analysis}

The Schnorr signature scheme is the application of the Fiat-Shamir transform to the Schnorr identification scheme.

Unlike Construction 13.12 of~\cite[p. 482]{katz2020}, \texttt{veil.schnorr} transmits the commitment point $I$ as part
of the signature and the verifier calculates $I'$ vs transmitting the challenge scalar $r$ and calculating $r'$.
In this way, \texttt{veil.schnorr} is closer to EdDSA~\cite{brendel2021} or the Schnorr variant proposed by Hamburg
in~\cite{hamburg2017}.

\subsection{UF-CMA Security}\label{subsec:veil.schnorr-uf-cma}

Per Theorem 13.10 of~\cite[p. 478]{katz2020}, this construction is UF-CMA secure if the Schnorr identification scheme
is secure and the hash function is secure:

\begin{displayquote}
    Let $\Pi$ be an identification scheme, and let $\Pi'$ be the signature scheme that results by applying the
    Fiat-Shamir transform to it.
    If $\Pi$ is secure and $H$ is modeled as a random oracle, then $\Pi'$ is secure.
\end{displayquote}

Per Theorem 13.11 of~\cite[p. 481]{katz2020}, the security of the Schnorr identification scheme is conditioned on the
hardness of the discrete logarithm problem:

\begin{displayquote}
    If the discrete-logarithm problem is hard relative to $\mathcal{G}$, then the Schnorr identification scheme is
    secure.
\end{displayquote}

Per~\cite[Sec. 5.10]{bertoni2011sponge}, Cyclist is a suitable random oracle if the underlying permutation is
indistinguishable from a random permutation.
Thus, \texttt{veil.schnorr} is UF-CMA if the discrete-logarithm problem is hard relative to P-256 and Keccak-\emph{p} is
indistinguishable from a random permutation.

\subsection{sUF-CMA Security}\label{subsec:veil.schnorr-suf-cma}

Some Schnorr/EdDSA implementations (e.g.\ Ed25519) suffer from malleability issues, allowing for multiple valid
signatures for a given signer and message~\cite{brendel2021}.
Chalkias et al.~\cite{chalkias2020} describe a strict verification function for Ed25519 which achieves sUF-CMA security
in addition to strong binding:

\begin{displayquote}
    \begin{enumerate}
        \item Reject the signature if $S \not\in \{0,\ldots,L-1\}$.
        \item Reject the signature if the public key $A$ is one of 8 small order points.
        \item Reject the signature if $A$ or $R$ are non-canonical.
        \item Compute the hash $\text{SHA2}_{512}(R||A||M)$ and reduce it mod $L$ to get a scalar $h$.
        \item Accept if $8(S \cdot B)-8R-8(h \cdot A)=0$.
    \end{enumerate}
\end{displayquote}

Rejecting $S \geq L$ makes the scheme sUF-CMA secure, and rejecting small order $A$ values makes the scheme strongly
binding.
\texttt{veil.schnorr}'s use of canonical point and scalar encoding routines obviate the need for these checks.
Likewise, P-256 is a prime order group, which obviates the need for cofactoring in verification.

When implemented with a prime order group and canonical encoding routines, the Schnorr signature scheme is strongly
unforgeable under chosen message attack (sUF-CMA) in the random oracle model~\cite{pointcheval2000} and even with
practical cryptographic hash functions~\cite{neven2009}.

\subsection{Key Privacy}\label{subsec:veil.schnorr-key-privacy}

The EdDSA variant (i.e.\ $S=(I,s)$) is used over the traditional Schnorr construction (i.e.\ $S=(r,s)$) to enable the
variable-time computation of $I'=[s]G - [r]Q$, which provides a ~30\% performance improvement.
That construction, however, allows for the recovery of the signing public key given a signature and a message: given the
commitment point $I$, one can calculate $Q=-[r^{-1}](I - [s]G)$.

For Veil, this behavior is not desirable.
A global passive adversary should not be able to discover the identity of a signer from a signed message.

To eliminate this possibility, \texttt{veil.schnorr} encrypts both components of the signature with a duplex keyed with
the signer's public key in addition to the message.
An attack which recovers the plaintext of either signature component in the absence of the public key would imply that
Cyclist is not IND-CPA\@.

\subsection{Indistinguishability From Random Noise}\label{subsec:veil.schnorr-indistinguishability}

Given that both signature components are encrypted with Cyclist, an attack which distinguishes between a
\texttt{veil.schnorr} and random noise would also imply that Cyclist is not IND-CPA\@.
