\section{Construction Techniques}\label{sec:construction-techniques}

Veil uses a few common construction techniques in its design which bear specific mention.

\subsection{Unkeyed And Keyed Duplexes}\label{subsec:cons-keyed-duplexes}

Veil uses Cyclist, which offers both unkeyed (``hash'') and keyed modes.
All Veil constructions begin in the unkeyed mode by absorbing a constant domain separation string (e.g.\
\texttt{veil.mres}).
To convert from an unkeyed duplex to a keyed duplex, a 512-bit key is derived from the unkeyed duplex's state and used
to initialize a keyed duplex:

\begin{algorithm}[ht]
    \caption{Converting an unkeyed Cyclist duplex to a keyed duplex.}
    \begin{algorithmic}
        \State \Call{Absorb}{\texttt{example.domain}}\Comment{Initialize an unkeyed duplex.}
        \State \Call{Absorb}{$X$}\Comment{Absorb some input data.}
        \State $K \gets $ \Call{SqueezeKey}{$64$}\Comment{Squeeze a 512-bit key.}
        \State \Call{Cyclist}{$K$, $\epsilon$, $\epsilon$}\Comment{Initialize a keyed duplex.}
        \State $C \gets $ \Call{Encrypt}{$P$}\Comment{Use the keyed duplex.}
    \end{algorithmic}
    \label{alg:duplex-convert}
\end{algorithm}

The unkeyed duplex is used as a kind of key derivation function, with the lower absorb rate of Cyclist's unkeyed mode
providing better avalanching properties.

\subsection{Integrated Constructions}\label{subsec:cons-integrated-constructions}

Cyclist is a cryptographic duplex, thus each operation is cryptographically dependent on the previous operations.
Veil makes use of this by integrating different types of constructions to produce a single, unified construction.
Instead of having to pass forward specific values (e.g.\ hashes of values or derived keys) to ensure cryptographic
dependency, Cyclist allows for constructions which simply absorb all values, thus ensuring transcript integrity of
complex protocols.

For example, a traditional hybrid encryption scheme like HPKE~\cite{rfc9180} will describe a key encapsulation mechanism
(KEM) like X25519 and a data encapsulation mechanism (DEM) like AES-GCM and link the two together via a key derivation
function (KDF) like HKDF by deriving a key and nonce for the DEM from the KEM output.

In contrast, the same construction using Cyclist would be the following three operations, in order (\ref{alg:hpke}):

\begin{algorithm}[ht]
    \caption{HPKE in Cyclist.}
    \begin{algorithmic}[1]
        \State \Call{Cyclist}{$[d_E]Q_R$, $\epsilon$, $\epsilon$}\label{alg:hpke:key}
        \State $C \gets $ \Call{Encrypt}{$P$}\label{alg:hpke:encrypt}
        \State $T \gets $ \Call{Squeeze}{$16$}\label{alg:hpke:tag}
    \end{algorithmic}
    \label{alg:hpke}
\end{algorithm}

The duplex is keyed with the shared secret point (line~\ref{alg:hpke:key}), used to encrypt the plaintext
(line~\ref{alg:hpke:encrypt}), and finally used to squeeze an authentication tag (line~\ref{alg:hpke:tag}).
Each operation modifies the duplex's state, making the final $\textsc{Squeeze}$ operation dependent on both the
previous $\textsc{Encrypt}$ operation (and its argument, $P$) but also the $\textsc{Cyclist}$ operation before it.

This is both a dramatically clearer way of expressing the overall hybrid public-key encryption construction and more
efficient: because the ephemeral shared secret point is unique, no nonce need be derived (or no all-zero nonce need be
justified in an audit).

\subsubsection{Process History As Hidden State}

A subtle but critical benefit of integrating constructions via a cryptographic duplex is that authenticators produced
via \textsc{Squeeze} operations are dependent on the entire process history of the duplex, not just on the emitted
ciphertext.
The DEM components of Alg.~\ref{alg:hpke} (i.e.\ \textsc{Encrypt}/\textsc{Squeeze}) are superficially similar to an
Encrypt-then-MAC ($\EtM$) construction, but where an adversary in possession of the MAC key can forge authenticators
given an $\EtM$ ciphertext, the duplex-based approach makes that infeasible.
The output of the \textsc{Squeeze} operation is dependent not just on the keying material (i.e.\ the \textsc{Cyclist}
operation) but also on the plaintext $P$.
An adversary attempting to forge an authenticator given only key material and ciphertext will be unable to reconstruct
the duplex's state and thus unable to compute their forgery.

\subsection{Hedged Ephemeral Values}\label{subsec:cons-hedged-ephemeral-values}

When generating ephemeral values, Veil uses Aranha et al.'s ``hedged signature'' technique~\cite{aranha2020} to mitigate
against both catastrophic randomness failures and differential fault attacks against purely deterministic schemes.

Specifically, the duplex's state is cloned, and the clone absorbs a context-specific secret value (e.g.\ the signer's
private key in a digital signature scheme) and a 64-byte random.
The clone duplex is used to produce the ephemeral value or values for the scheme.

For example, the following operations would be performed on the cloned duplex (\ref{alg:hedged-ephemeral}):

\begin{algorithm}[ht]
    \caption{Hedged ephemeral generation with Cyclist.}
    \begin{algorithmic}[0]
        \Clone \Comment{Clone the duplex's state.}
        \State \Call{Absorb}{$d$}\Comment{Absorb a private key.}
        \State $v \rgets \allbits{512}$\Comment{Generate a random value.}
        \State \Call{Absorb}{$v$}\Comment{Absorb the random value.}
        \State $x \gets$ \Call{Squeeze}{$64$} $\modl$\Comment{Squeeze a hedged ephemeral scalar.}
        \State \textbf{yield} $x$\Comment{Return $x$ to the outer context.}
        \End \Comment{Destroy the cloned duplex's state.}
    \end{algorithmic}
    \label{alg:hedged-ephemeral}
\end{algorithm}

The ephemeral scalar $x$ is returned to the context of the original construction and the cloned duplex is discarded.
This ensures that even in the event of a catastrophic failure of the random number generator, $x$ is still unique
relative to $d$.
Depending on the uniqueness needs of the construction, an ephemeral value can be hedged with a plaintext in addition to
a private key.
