\section{Encrypted Headers}\label{sec:veil.sres}

\texttt{veil.sres} implements a single-receiver, deniable signcryption scheme which Veil uses to
encrypt message headers.
It integrates an ephemeral ECDH KEM, a Cyclist DEM, and a designated-verifier Schnorr signature
scheme to provide multi-user insider security.

\subsection{Encrypting A Header}\label{subsec:veil.sres-encrypt}

Encrypting a header is described in full in Alg.~\ref{alg:veil.sres-encrypt}\@.

\begin{algorithm}
    \caption{Encrypting a header with sender's key pair $(d_S, Q_S)$, ephemeral key pair
        $(d_E, Q_E)$, receiver's public key $Q_R$, nonce $N$, and plaintext $P$.}
    \begin{algorithmic}[1]
        \Function{EncryptHeader}{$(d_S, Q_S), (d_E, Q_E), Q_R, N, P$}
        \State \Call{Absorb}{\texttt{veil.sres}}\Comment{Initialize an unkeyed duplex.}
        \State \Call{Absorb}{$Q_S$}\label{alg:veil.sres-encrypt-bind-sender}\Comment{Absorb the sender's public key.}
        \State \Call{Absorb}{$Q_R$}\Comment{Absorb the receiver's public key.}
        \State \Call{Absorb}{$N$}\Comment{Absorb the nonce.}
        \State \Call{Absorb}{$d_S[Q_R]$}\Comment{Absorb the static ECDH shared secret.}
        \State \Call{Cyclist}{\textsc{SqueezeKey}($64$), $\epsilon$, $\epsilon$}\Comment{Convert to a keyed duplex.}
        \State
        \State $C_0 \gets$ \Call{Encrypt}{$Q_E$}\Comment{Encrypt the ephemeral public key.}
        \State \Call{Absorb}{$d_E[Q_R]$}\Comment{Absorb the ephemeral ECDH shared secret.}
        \State $C_1 \gets$ \Call{Encrypt}{$P$}\Comment{Encrypt the plaintext.}
        \State
        \Clone \Comment{Clone the duplex's state.}
        \State \Call{Absorb}{$d_S$}\Comment{Absorb the sender's private key.}
        \State $v \rgets \allbits{512}$\Comment{Generate a random value.}
        \State \Call{Absorb}{$v$}\Comment{Absorb the random value.}
        \State $k \gets$ \Call{Squeeze}{$64$} $\modl$\Comment{Squeeze a commitment scalar.}
        \State \textbf{yield} $k$
        \End
        \State
        \State $I \gets [k]G$\Comment{Calculate the commitment point.}
        \State $S_0 \gets$ \Call{Encrypt}{$I$}\Comment{Encrypt the commitment point.}
        \State
        \State $r \gets$ \Call{Squeeze}{$64$} $\modl$
        \label{alg:veil.sres-encrypt-challenge}
        \Comment{Squeeze a challenge scalar.}
        \State $s \gets d_S{r}+k$\Comment{Calculate the proof scalar.}
        \State
        \State $X \gets [s]Q_R$\Comment{Calculate the proof point.}
        \State $S_1 \gets$ \Call{Encrypt}{$X$}\Comment{Encrypt the proof point.}
        \State
        \State \textbf{return} $C_0||C_1||S_0||S_1$
        \EndFunction
    \end{algorithmic}
    \label{alg:veil.sres-encrypt}
\end{algorithm}

\subsection{Decrypting A Header}\label{subsec:veil.sres-decrypt}

Decrypting a header is described in full in Alg.~\ref{alg:veil.sres-decrypt}\@.

\begin{algorithm}
    \caption{Decrypting a header with receiver's key pair $(d_R, Q_R)$, sender's public key $Q_S$,
        nonce $N$, and ciphertext $C_0||C_1||S_0||S_1$.}
    \begin{algorithmic}[1]
        \Function{DecryptHeader}{$(d_R, Q_R), Q_S, N, C_0||C_1||S_0||S_1$}
        \State \Call{Absorb}{\texttt{veil.sres}}\Comment{Initialize an unkeyed duplex.}
        \State \Call{Absorb}{$Q_S$}\label{alg:veil.sres-decrypt-bind-sender}\Comment{Absorb the sender's public key.}
        \State \Call{Absorb}{$Q_R$}\Comment{Absorb the receiver's public key.}
        \State \Call{Absorb}{$N$}\Comment{Absorb the nonce.}
        \State
        \State \Call{Absorb}{$d_R[Q_S]$}\Comment{Absorb the static ECDH shared secret.}
        \State \Call{Cyclist}{\textsc{SqueezeKey}($64$), $\epsilon$, $\epsilon$}\Comment{Convert to a keyed duplex.}
        \State
        \State ${Q_E}' \gets$ \Call{Decrypt}{$C_0$}\Comment{Decrypt the ephemeral public key.}
        \State \Call{Absorb}{$d_R[{Q_E}']$}\Comment{Absorb the ephemeral ECDH shared secret.}
        \State $P' \gets$ \Call{Decrypt}{$C_1$}\Comment{Decrypt the plaintext.}
        \State
        \State $I \gets$ \Call{Decrypt}{$S_0$}\Comment{Decrypt the commitment point.}
        \State $r' \gets$ \Call{Squeeze}{$64$} $\modl$\Comment{Squeeze a challenge scalar.}
        \State
        \State $X \gets$ \Call{Decrypt}{$S_1$}\Comment{Decrypt the proof point.}
        \State $X' \gets [d_R](I + [r']Q_S)$\Comment{Re-calculate the proof point.}
        \State
        \If{$X' \checkeq X$}
        \Comment{Ensure the ciphertext is authentic.}
        \State \textbf{return} ${Q_E}', P'$
        \Else
        \State \textbf{return} $\bot$
        \EndIf
        \EndFunction
    \end{algorithmic}
    \label{alg:veil.sres-decrypt}
\end{algorithm}

\subsection{Constructive Analysis Of \texttt{veil.sres}}\label{subsec:veil.sres-analysis}

\texttt{veil.sres} is an integration of two well-known constructions: an ECIES-style hybrid public
key encryption scheme and a designated-verifier Schnorr signature scheme.

The initial portion of \texttt{veil.sres} is equivalent to ECIES (see Construction 12.23 of~\cite[p.
    435]{katz2020}), \@(with the commitment point $I$ as an addition to the ciphertext, and the
challenge scalar $r$ serving as the authentication tag for the data encapsulation mechanism) and is
IND-CCA2 secure (see Corollary 12.14 of~\cite[p. 436]{katz2020}).

The latter portion of \texttt{veil.sres} is a designated-verifier Schnorr signature scheme which
adapts an EdDSA-style Schnorr signature scheme by multiplying the proof scalar $s$ by the receiver's
public key $Q_R$ to produce a designated-verifier point $X$~\cite{steinfeld2004}.
The EdDSA-style Schnorr signature is sUF-CMA secure when implemented in a prime order group and a
cryptographic hash function~\cite{brendel2021, chalkias2020, pointcheval2000, neven2009} (see also
Sec.~\ref{sec:veil.schnorr})\@.

\subsection{Multi-User Confidentiality}\label{subsec:veil.sres-conf}

One of the two main goals of the \texttt{veil.sres} is confidentiality in the multi-user setting
(see Sec.~\ref{subsec:sec-conf}), or the inability of an adversary $\Adversary$ to learn information
about plaintexts.

\subsubsection{Outsider Confidentiality}

First, we evaluate the confidentiality of \texttt{veil.sres} in the multi-user outsider setting (see
Sec.~\ref{subsubsec:sec-conf-outsider}), in which the adversary $\Adversary$ knows the public keys
of all users but none of their private keys~\cite[p. 44]{baek2010}\@.

The classic multi-user attack on the generic Encrypt-Then-Sign ($\EtS$) construction sees
$\Adversary$ strip the signature $\sigma$ from the challenge ciphertext
\[
    C=(c,\sigma,Q_S,Q_R)
\]
and replace it with
\[
    \sigma' \rgets \text{Sign}(d_{\Adversary},c)
\]
to produce an attacker ciphertext
\[
    C'=(c,\sigma',Q_{\Adversary},Q_R)
\]
at which point $\Adversary$ can trick the receiver into decrypting the result and giving
$\Adversary$ to the randomly-chosen plaintext $m_0 \lor m_1$~\cite[p. 40]{an2010}.
This attack is not possible with \texttt{veil.sres}\@, as the sender's public key is strongly bound
during encryption (see Alg.~\ref{alg:veil.sres-encrypt},
Line~\ref{alg:veil.sres-encrypt-bind-sender}) and decryption (see Alg.~\ref{alg:veil.sres-decrypt},
Line~\ref{alg:veil.sres-decrypt-bind-sender}).

$\Adversary$ is unable to forge valid signatures for existing ciphertexts, limiting them to passive
attacks. A passive attack on any of the three components of \texttt{veil.sres} ciphertexts--$C$,
$S_0$, $S_1$--would only be possible if Cyclist is not IND-CPA secure.

Therefore, \texttt{veil.sres} provides confidentiality in the multi-user outsider setting.

\subsubsection{Insider Confidentiality}

Next, we evaluate the confidentiality of \texttt{veil.sres} in the multi-user insider setting (see
Sec.~\ref{subsubsec:sec-conf-insider}), in which the adversary $\Adversary$ knows the sender's
private key in addition to the public keys of both users~\cite[p. 45--46]{baek2010}\@.

$\Adversary$ cannot decrypt the message by themselves, as they do not know either $d_E$ or $d_R$ and
cannot calculate the ECDH shared secret $[d_E]Q_R=[d_R]Q_E=[d_E{d_R}G]$.

$\Adversary$ also cannot trick the receiver into decrypting an equivalent message by replacing the
signature, despite $\Adversary\text{'s}$ ability to use $d_S$ to create new signatures.
In order to generate a valid signature on a ciphertext $c'$ (e.g.\ $c'=c||1$), $\Adversary$ would
have to squeeze a valid challenge scalar $r'$ from the duplex state (see
Alg.~\ref{alg:veil.sres-encrypt}, Line~\ref{alg:veil.sres-encrypt-challenge}).
Unlike the signature hash function in the generic $\EtS$ composition, however, the duplex state is
cryptographically dependent on values $\Adversary$ does not know, specifically the ECDH shared
secret $[d_E]Q_S$ \@(via the \textsc{Absorb} operation) and the plaintext $P$ (via the
\textsc{Encrypt} operation).

Therefore, \texttt{veil.sres} provides confidentiality in the multi-user insider setting.

\subsection{Multi-User Authenticity}\label{subsec:veil.sres-auth}

The second of the two main goals of the \texttt{veil.sres} is authenticity in the multi-user setting
(see Sec.~\ref{subsec:sec-auth}), or the inability of an adversary $\Adversary$ to forge valid
ciphertexts.

\subsubsection{Outsider Authenticity}

First, we evaluate the authenticity of \texttt{veil.sres} in the multi-user outsider setting (see
Sec.~\ref{subsubsec:sec-auth-outsider}), in which the adversary $\Adversary$ knows the public keys
of all users but none of their private keys~\cite[p. 47]{baek2010}\@.

Because the Schnorr signature scheme is sUF-CMA secure, it is infeasible for $\Adversary$ to forge a
signature for a new message or modify an existing signature for an existing message.
Therefore, \texttt{veil.sres} provides authenticity in the multi-user outsider setting.

\subsubsection{Insider Authenticity}

Next, we evaluate the authenticity of \texttt{veil.sres} in the multi-user insider setting (see
Sec.~\ref{subsubsec:sec-auth-insider}), in which the adversary $\Adversary$ knows the receiver's
private key in addition to the public keys of both users~\cite[p. 48]{baek2010}\@.

Again, the Schnorr signature scheme is sUF-CMA secure and the signature is created using the
signer's private key.
The receiver (or $\Adversary$ in possession of the receiver's private key) cannot forge signatures
for new messages. Therefore, \texttt{veil.sres} provides authenticity in the multi-user insider
setting.

\subsection{Limited Deniability}\label{subsec:veil.sres-deniability}

\texttt{veil.sres}{'s} use of a designated-verifier Schnorr scheme provides limited deniability for
senders (see Sec.~\ref{subsec:security-deniability}).
Without revealing $d_R$, the receiver cannot prove the authenticity of a message \@(including the
identity of its sender) to a third party.

\subsection{Indistinguishability From Random Noise}\label{subsec:veil.sres-indistinguishability}

All of the components of a \texttt{veil.sres} ciphertext--$C$, $S_0$, and $S_1$--are Cyclist
ciphertexts.
An adversary in the outsider setting \@(i.e.\ knowing only public keys) is unable to calculate any
of the key material used to produce the ciphertexts;
a distinguishing attack is infeasible if Cyclist is IND-CPA secure.

\subsection{Re-use Of Ephemeral Keys}\label{subsec:veil.sres-re-using-ephemeral-keys}

The re-use of an ephemeral key pair $(d_E, Q_E)$ across multiple ciphertexts does not impair the
confidentiality of the scheme provided $(N, Q_R)$ pairs are not re-used~\cite{bellare2003}\@.
An adversary who compromises a retained ephemeral private key would be able to decrypt all messages
the sender encrypted using that ephemeral key, thus the forward sender security is bounded by the
sender's retention of the ephemeral private key.
