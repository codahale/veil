\documentclass{article}

\usepackage[utf8]{inputenc}
\usepackage{amsmath}
\usepackage{amssymb}
\usepackage{polyglossia}
\usepackage{csquotes}
\usepackage{xpatch}
\usepackage[style=trad-abbrv]{biblatex}
\usepackage[
    breaklinks=true,
    bookmarks=true,
    colorlinks=true,
    citecolor=blue,
    urlcolor=blue,
    linkcolor=blue,
    pdfborder={0 0 0}
]{hyperref}
\usepackage{url}
\usepackage{algpseudocode}
\usepackage{algorithm}
\usepackage{algorithmicx}

\setdefaultlanguage{english}
\addbibresource[]{design.bib}

\title{The Veil Cryptosystem}
\author{Coda Hale}
\date{\today}

% common notation
\newcommand{\modl}{\bmod~\ell}
\newcommand{\allbits}[1]{\{0,1\}^{#1}}
\newcommand{\rgets}[0]{\stackrel{\$}{\gets}}
\newcommand{\checkeq}[0]{\stackrel{?}{=}}

% security model shorthand
\newcommand{\EtS}[0]{\mathcal{E}t\mathcal{S}}
\newcommand{\EtM}[0]{\mathcal{E}t\mathcal{M}}
\newcommand{\StE}[0]{\mathcal{S}t\mathcal{E}}
\newcommand{\Adversary}[0]{\mathcal{A}}

% duplex stuff
\algdef{SE}[CLONE]{Clone}{End}{\textbf{with clone do}}[0]{\algorithmicend}%

\begin{document}
    \clearpage
    \phantomsection
    \pdfbookmark[1]{Title}{title}
    \maketitle
    \begin{abstract}
        Veil is a public-key cryptosystem that provides confidentiality, authenticity, and integrity services for
        messages of arbitrary sizes and multiple receivers.
        This document describes its cryptographic constructions, their security properties, and how they are combined to
        implement Veil's feature set.
    \end{abstract}

    \clearpage
    \phantomsection
    \pdfbookmark[1]{\contentsname}{contents}
    \tableofcontents

    \clearpage
    \phantomsection
    \pdfbookmark[2]{Algorithms}{algorithms}
    \listofalgorithms

    \section{Motivation}\label{sec:motivation}

Veil is a clean-slate effort to build a secure, asynchronous, PGP-like messaging cryptosystem using modern tools and
techniques to resist new attacks from modern adversaries.
PGP provides confidential and authentic multi-receiver messaging, but with many deficiencies.

\subsection{Cryptographic Agility}\label{subsec:cryptographic-agility}

PGP~was initially released in 1991 using a symmetric algorithm called BassOmatic invented by Phil
Zimmerman himself.
Since then, it's supported IDEA, DES, Triple-DES, CAST5, Blowfish,
SAFER-SK128, AES-128, AES-192, AES-256, Twofish, and Camellia,
with proposed support for ChaCha20.
For hash algorithms, it's supported MD5, SHA-1, RIPE-MD160, MD2, ``double-width SHA'', TIGER/192, HAVAL, SHA2-256,
SHA2-384, SHA2-512, SHA2-224.
For public key encryption, it's supported RSA, ElGamal, Diffie-Hellman, and ECDH, all with different parameters.
For digital signatures, it's supported RSA, DSA, ElGamal, ECDSA, and EdDSA, again, all with different parameters.

As \textcite{langley2016} said regarding TLS:

\begin{displayquote}
    Cryptographic agility is a huge cost.
    Implementing and supporting multiple algorithms means more code.
    More code begets more bugs.
    More things in general means less academic focus on any one thing, and less testing and code-review per thing.
    Any increase in the number of options also means more combinations and a higher chance for a bad interaction to
    arise.
\end{displayquote}

At best, each of these algorithms represents a geometrically increasing burden on implementors, analysts, and users.
At worst, they represent a catastrophic risk to the security of the system~\cite{nguyen2004, blessing2021}\@.

A modern system would use a limited number of cryptographic primitives and use a single instance of each.

\subsection{Informal Constructions}\label{subsec:informal-constructions}

PGP messages use a Sign-Then-Encrypt ($\StE$) construction, which is insecure given an encryption
oracle~\cite[p. 41]{an2010}\@:

\begin{displayquote}
    In the $\StE$ scheme, the adversary $\Adversary$ can easily break the sUF-CMA security in the outsider model.
    It can ask \@[the encryption oracle] to signcrypt a message $m$ for $R'$ and get
    $C=(\textsc{Encrypt}(pk_{R'},m||\sigma),ID_S,ID_{R'})$, where $\sigma \stackrel{R}{\gets} \textsc{Sign}(pk_S,m)$.
    Then, it can recover $m||\sigma$ using $sk_{R'}$ and forge the signcryption ciphertext
    $C' = (\textsc{Encrypt}(pk_R,m||\sigma),ID_S,ID_R)$.
\end{displayquote}

This may seem like an academic distinction, but this attack is trivial to mount.
If you send your boss an angry resignation letter signed and encrypted with PGP, your boss can re-transmit that to your
future boss, encrypted with her public key.

A modern system would use established, analyzed constructions with proofs in established models to achieve established
notions with reasonable reductions to weak assumptions.

\subsection{Non-Repudiation}\label{subsec:non-repudiation}

A standard property of digital signatures is that of \emph{non-repudiation}\@, or the inability of the signer to deny
they signed a message.
Any possessor of the signer's public key, a message, and a signature can verify the signature for themselves.
For explicitly signed, public messages, this is a very desirable property.
For encrypted, confidential messages, this is not.

Similar to the vindictive boss scenario above, an encrypted-then-signed PGP message can be decrypted by an intended
receiver \@(or someone in possession of their private key) and presented to a third party as an unencrypted, signed
message without having to reveal anything about themselves.
The inability of PGP to preserve the privacy context of confidential messages should rightfully have a chilling effect
on its users~\cite{borisov2004}\@.

A modern system would be designed to provide some level of deniability to confidential messages.

\subsection{Global Passive Adversaries}\label{subsec:global-passive-adversaries}

A new type of adversary which became immediately relevant to the post-Snowden era is the Global Passive Adversary, which
monitors all traffic on all links of a network.
For an adversary with an advantaged network position (e.g.\ a totalitarian state), looking for
cryptographically-protected messages is trivial given the metadata they often expose.
Even privacy features like GnuPG's \texttt{--hidden-recipients} still produce encrypted messages which are trivially
identifiable as encrypted messages, because PGP messages consist of packets with explicitly identifiable metadata.
In addition to being secure, privacy-enhancing technologies must be undetectable.

\textcite{bernstein2013} summarized this dilemma:

\begin{displayquote}
    Cryptography hides patterns in user data but does not evade censorship if the censor can recognize patterns in the
    cryptography itself.
\end{displayquote}

A modern system would produce messages without recognizable metadata or patterns.

    \section{Security Model And Notions}\label{sec:security-model-and-notions}

\subsection{Multi-User Confidentiality}\label{subsec:sec-conf}

To evaluate the confidentiality of a scheme, we consider an adversary $\Adversary$ attempting to attack a sender and
receiver~\cite[p. 44]{baek2010}\@.
$\Adversary$ creates two equal-length messages $(m_0, m_1)$, the sender selects one at random and encrypts it, and
$\Adversary$ guesses which of the two has been encrypted without tricking the receiver into decrypting it for them.
To model real-world possibilities, we assume $\Adversary$ has three capabilities:

\begin{enumerate}
    \item $\Adversary$ can create their own key pairs.
    Veil does not have a centralized certificate authority and creating new key pairs is intentionally trivial.
    \item $\Adversary$ can trick the sender into encrypting arbitrary plaintexts with arbitrary public keys.
    This allows us to model real-world flaws such as servers which return encrypted error messages with client-provided
    data~\cite{yu2004}\@.
    \item $\Adversary$ can trick the receiver into decrypting arbitrary ciphertexts from arbitrary senders.
    This allows us to model real-world flaws such as padding oracles~\cite{rizzo2010practical}\@.
\end{enumerate}

Given these capabilities, $\Adversary$ can mount an attack in two different settings: the outsider setting and the
insider setting.

\subsubsection{Outsider Confidentiality}\label{subsubsec:sec-conf-outsider}

In the multi-user outsider model, we assume $\Adversary$ knows the public keys of all users but none of their private
keys~\cite[p. 44]{baek2010}\@.

The multi-user outsider model is useful in evaluating the strength of a scheme against adversaries who have access to
some aspect of the sender and receiver's interaction with messages \@(e.g.\ a padding oracle) but who have not compromised
the private keys of either.

\subsubsection{Insider Confidentiality}\label{subsubsec:sec-conf-insider}

In the multi-user insider model, we assume $\Adversary$ knows the sender's private key in addition to the public keys of
both users~\cite[p. 45--46]{baek2010}\@.

The multi-user insider model is useful in evaluating the strength of a scheme against adversaries who have compromised
a user.

\paragraph{Forward Sender Security}

A scheme which provides confidentiality in the multi-user insider setting is called \emph{forward sender secure} because
an adversary who compromises a sender cannot read messages that sender has previously encrypted~\cite{canetti2003}\@.

\subsection{Multi-User Authenticity}\label{subsec:sec-auth}

To evaluate the authenticity of a scheme, we consider an adversary $\Adversary$ attempting to attack a sender and
receiver~\cite[p. 47]{baek2010}\@.
$\Adversary$ attempts to forge a ciphertext which the receiver will decrypt but which the sender never encrypted.
To model real-world possibilities, we again assume $\Adversary$ has three capabilities:

\begin{enumerate}
    \item $\Adversary$ can create their own key pairs.
    \item $\Adversary$ can trick the sender into encrypting arbitrary plaintexts with arbitrary public keys.
    \item $\Adversary$ can trick the receiver into decrypting arbitrary ciphertexts from arbitrary senders.
\end{enumerate}

As with multi-user confidentiality, this can happen in the outsider setting and the insider setting.

\subsubsection{Outsider Authenticity}\label{subsubsec:sec-auth-outsider}

In the multi-user outsider model, we again assume $\Adversary$ knows the public keys of all users but none of their
private keys~\cite[p. 47]{baek2010}\@.

Again, this is useful to evaluate the strength of a scheme in which $\Adversary$ has some insight into senders and
receivers but has not compromised either.

\subsubsection{Insider Authenticity}\label{subsubsec:sec-auth-insider}

In the multi-user insider model, we assume $\Adversary$ knows the receiver's private key in addition to the public keys
of both users~\cite[p. 48]{baek2010}\@.

\paragraph{Key Compromise Impersonation}

A scheme which provides authenticity in the multi-user insider setting effectively resists
\emph{key compromise impersonation}, in which $\Adversary$, given knowledge of a receiver's private key, can
forge messages to that receiver from arbitrary senders~\cite{strangio2006}\@.
The classic example is authenticated Diffie-Hellman (e.g.\ RFC 9180~\cite{rfc9180, alwen2021}), in which the static
Diffie-Hellman shared secret point $K=[d_S]Q_R$ is used to encrypt a message and its equivalent $K'=[d_R]Q_S$ is used
to decrypt it.
An attacker in possession of the receiver's private key $d_R$ and the sender's public key $Q_S$ can simply encrypt the
message using $K'=[d_R]Q_S$ without ever having knowledge of $d_S$.
Digital signatures are a critical element of schemes which provide insider authenticity, as they give receivers a way to
verify the authenticity of a message using authenticators they \@(or an adversary with their private key) could never
construct themselves.

\subsection{Insider vs. Outsider Security}\label{subsec:security-insider-vs-outsider}

The multi-receiver setting motivates a focus on insider security over the traditional emphasis on outsider security
(contra~\cite[p. 26]{an2010}\cite[p. 46]{baek2010}; see~\cite{badertscher2018}).
Given a probability of an individual key compromise $P$, a multi-user system of $N$ users has an overall $1-(1-P)^N$
probability of at least one key being compromised.
A system with an exponentially increasing likelihood of losing all confidentiality and authenticity properties is not
acceptable.

\subsection{Indistinguishable From Random Noise}\label{subsec:security-indistinguishable}

Indistinguishability from random noise is a critical property for censorship-resistant
communication~\cite{bernstein2013}\@:

\begin{displayquote}
    Censorship-circumvention tools are in an arms race against censors.
    The censors study all traffic passing into and out of their controlled sphere, and try to disable
    censorship-circumvention tools without completely shutting down the Internet.
    Tools aim to shape their traffic patterns to match unblocked programs, so that simple traffic profiling cannot
    identify the tools within a reasonable number of traces;
    the censors respond by deploying firewalls with increasingly sophisticated deep-packet inspection.

    Cryptography hides patterns in user data but does not evade censorship if the censor can recognize patterns in the
    cryptography itself.
\end{displayquote}

\subsection{Limited Deniability}\label{subsec:security-deniability}

The inability of a receiver (or an adversary in possession of a receiver's private key) to prove the authenticity of a
message to a third party is critical for privacy.
Other privacy-sensitive protocols achieve this by forfeiting insider authenticity or authenticity
altogether~\cite{borisov2004}\@.
Veil achieves a limited version of deniability: a receiver can only prove the authenticity of a message to
a third party by revealing their own private key.
This deters a dishonest receiver from selectively leaking messages and requires all-or-nothing disclosure from an
adversary who compromises an honest receiver.

    \section{Cryptographic Primitives}\label{sec:cryptographic-primitives}

In the interests of cryptographic minimalism, Veil uses just two distinct cryptographic primitives:

\begin{itemize}
    \item Keccyak-128~\cite{daemen2020,bertoni2018} for confidentiality, authentication, and integrity.
    \item Curve25519~\cite{bernstein2006}, with X25519~\cite{rfc7748} for key agreement and qDSA~\cite{renes2017} for
          authenticity.
\end{itemize}

\subsection{Keccyak-128}\label{subsec:keccyak}

Keccyak-128 is the adaptation of the Cyclist duplex construction from Xoodyak~\cite{daemen2020} to the
Keccak-\emph{p}[1600,12] permutation instead of Xoodoo (thus ``Keccyak'').

Cyclist is a permutation-based cryptographic duplex, a cryptographic primitive that provides symmetric-key
confidentiality, integrity, and authentication via a single object~\cite{daemen2020}.
Duplexes offer a way to replace complex, ad-hoc constructions combining encryption algorithms, cipher modes,
AEADs, MACs, and hash algorithms using a single primitive~\cite{daemen2020, bertoni2011duplex}.
Duplexes have security properties which reduce to the properties of the cryptographic sponge, which themselves reduce to
the strength of the underlying permutation~\cite{bertoni2008}.

The Keccak-\emph{p}[1600,12] permutation is the basis of the KangarooTwelve hash algorithm~\cite{bertoni2018} and allows
for much higher throughput in software than the Xoodoo permutation at the expense of requiring a larger state.
It has a 1600-bit width, like Keccak-\emph{f}[1600] (the basis of SHA-3), but with a reduced number of rounds for speed.
Xoodyak's Cyclist parameters of $R_\text{hash}=b-256$, $R_\text{kin}=b-32$, $R_\text{kout}=b-192$, and
$\ell_\text{ratchet}=16$ are adapted for the larger permutation width of $b=1600$.
The resulting construction targets a 128-bit security level and is $\sim$4x faster than SHA-256, $\sim$5x faster
than ChaCha20Poly1305, and $\sim$7.5x faster than AES-128-GCM in software.

Veil's security assumes that Cyclist's \textsc{Encrypt} operation is IND-CPA secure, its
\textsc{Squeeze} operation is sUF-CMA secure, and its
\textsc{Encrypt}/\textsc{Squeeze}-based authenticated encryption construction is IND-CCA2 secure.

\subsection{Curve25519}\label{subsec:curve25519}

Curve25519 uses a safe curve, is highly performant, targets a 128-bit security level, lends itself to constant-time
implementations, and can run in constrained environments~\cite{bernstein2006}.

X25519~\cite{rfc7748} (i.e. Diffie-Hellman) and qDSA~\cite{renes2017} provide for key agreement and authenticity
services using only the Montgomery curve.

Veil's security assumes that the Gap Discrete Logarithm and Gap Diffie-Hellman problems are hard relative to
Curve25519.

    \section{Construction Techniques}\label{sec:construction-techniques}

Veil uses a few common construction techniques in its design which bear specific mention.

\subsection{Unkeyed And Keyed Duplexes}\label{subsec:cons-keyed-duplexes}

Veil uses Cyclist, which offers both unkeyed (``hash'') and keyed modes.
All Veil constructions begin in the unkeyed mode by absorbing a constant domain separation string (e.g.\
\texttt{veil.mres}).
To convert from an unkeyed duplex to a keyed duplex, a 512-bit key is derived from the unkeyed duplex's state and used
to initialize a keyed duplex:

\begin{algorithm}[ht]
    \caption{Converting an unkeyed Cyclist duplex to a keyed duplex.}
    \begin{algorithmic}
        \State \Call{Absorb}{\texttt{example.domain}}\Comment{Initialize an unkeyed duplex.}
        \State \Call{Absorb}{$X$}\Comment{Absorb some input data.}
        \State $K \gets $ \Call{SqueezeKey}{$64$}\Comment{Squeeze a 512-bit key.}
        \State \Call{Cyclist}{$K$, $\epsilon$, $\epsilon$}\Comment{Initialize a keyed duplex.}
        \State $C \gets $ \Call{Encrypt}{$P$}\Comment{Use the keyed duplex.}
    \end{algorithmic}
    \label{alg:duplex-convert}
\end{algorithm}

The unkeyed duplex is used as a kind of key derivation function, with the lower absorb rate of Cyclist's unkeyed mode
providing better avalanching properties.

\subsection{Integrated Constructions}\label{subsec:cons-integrated-constructions}

Cyclist is a cryptographic duplex, thus each operation is cryptographically dependent on the previous operations.
Veil makes use of this by integrating different types of constructions to produce a single, unified construction.
Instead of having to pass forward specific values (e.g.\ hashes of values or derived keys) to ensure cryptographic
dependency, Cyclist allows for constructions which simply absorb all values, thus ensuring transcript integrity of
complex protocols.

For example, a traditional hybrid encryption scheme like HPKE~\cite{rfc9180} will describe a key encapsulation mechanism
(KEM) like X25519 and a data encapsulation mechanism (DEM) like AES-GCM and link the two together via a key derivation
function (KDF) like HKDF by deriving a key and nonce for the DEM from the KEM output.

In contrast, the same construction using Cyclist would be the following three operations, in order (\ref{alg:hpke}):

\begin{algorithm}[ht]
    \caption{HPKE in Cyclist.}
    \begin{algorithmic}[1]
        \State \Call{Cyclist}{$[d_E]Q_R$, $\epsilon$, $\epsilon$}\label{alg:hpke:key}
        \State $C \gets $ \Call{Encrypt}{$P$}\label{alg:hpke:encrypt}
        \State $T \gets $ \Call{Squeeze}{$16$}\label{alg:hpke:tag}
    \end{algorithmic}
    \label{alg:hpke}
\end{algorithm}

The duplex is keyed with the shared secret point (line~\ref{alg:hpke:key}), used to encrypt the plaintext
(line~\ref{alg:hpke:encrypt}), and finally used to squeeze an authentication tag (line~\ref{alg:hpke:tag}).
Each operation modifies the duplex's state, making the final $\textsc{Squeeze}$ operation dependent on both the
previous $\textsc{Encrypt}$ operation (and its argument, $P$) but also the $\textsc{Cyclist}$ operation before it.

This is both a dramatically clearer way of expressing the overall hybrid public-key encryption construction and more
efficient: because the ephemeral shared secret point is unique, no nonce need be derived (or no all-zero nonce need be
justified in an audit).

\subsubsection{Process History As Hidden State}

A subtle but critical benefit of integrating constructions via a cryptographic duplex is that authenticators produced
via \textsc{Squeeze} operations are dependent on the entire process history of the duplex, not just on the emitted
ciphertext.
The DEM components of Alg.~\ref{alg:hpke} (i.e.\ \textsc{Encrypt}/\textsc{Squeeze}) are superficially similar to an
Encrypt-then-MAC ($\EtM$) construction, but where an adversary in possession of the MAC key can forge authenticators
given an $\EtM$ ciphertext, the duplex-based approach makes that infeasible.
The output of the \textsc{Squeeze} operation is dependent not just on the keying material (i.e.\ the \textsc{Cyclist}
operation) but also on the plaintext $P$.
An adversary attempting to forge an authenticator given only key material and ciphertext will be unable to reconstruct
the duplex's state and thus unable to compute their forgery.

\subsection{Hedged Ephemeral Values}\label{subsec:cons-hedged-ephemeral-values}

When generating ephemeral values, Veil uses Aranha et al.'s ``hedged signature'' technique~\cite{aranha2020} to mitigate
against both catastrophic randomness failures and differential fault attacks against purely deterministic schemes.

Specifically, the duplex's state is cloned, and the clone absorbs a context-specific secret value (e.g.\ the signer's
private key in a digital signature scheme) and a 64-byte random.
The clone duplex is used to produce the ephemeral value or values for the scheme.

For example, the following operations would be performed on the cloned duplex (\ref{alg:hedged-ephemeral}):

\begin{algorithm}[ht]
    \caption{Hedged ephemeral generation with Cyclist.}
    \begin{algorithmic}[0]
        \Clone \Comment{Clone the duplex's state.}
        \State \Call{Absorb}{$d$}\Comment{Absorb a private key.}
        \State $v \rgets \allbits{512}$\Comment{Generate a random value.}
        \State \Call{Absorb}{$v$}\Comment{Absorb the random value.}
        \State $x \gets$ \Call{Squeeze}{$32$} $\modl$\Comment{Squeeze a hedged ephemeral scalar.}
        \State \textbf{yield} $x$\Comment{Return $x$ to the outer context.}
        \End \Comment{Destroy the cloned duplex's state.}
    \end{algorithmic}
    \label{alg:hedged-ephemeral}
\end{algorithm}

The ephemeral scalar $x$ is returned to the context of the original construction and the cloned duplex is discarded.
This ensures that even in the event of a catastrophic failure of the random number generator, $x$ is still unique
relative to $d$.
Depending on the uniqueness needs of the construction, an ephemeral value can be hedged with a plaintext in addition to
a private key.

    \section{Digital Signatures}\label{sec:veil.schnorr}

\texttt{veil.schnorr} implements a Schnorr digital signature scheme.

\subsection{Signing A Message}\label{subsec:veil.schnorr-sign}

Signing a message is described in full in Alg.~\ref{alg:veil.schnorr-sign}.

\begin{algorithm}[!htp]
    \caption{
        Signing a message $M$ with a key pair $(d, Q)$.
    }
    \begin{algorithmic}
        \Function{Sign}{$(d_S, Q_S), M$}
            \State \Call{Absorb}{\texttt{veil.schnorr}}\Comment{Initialize an unkeyed duplex.}
            \State \Call{Absorb}{$Q$}\Comment{Absorb the signer's public key.}
            \State \Call{Absorb}{$M$}\Comment{Absorb the message.}
            \State
            \Clone \Comment{Clone the duplex's state.}
            \State \Call{Absorb}{$d$}\Comment{Absorb the sender's private key.}
            \State $v \rgets \allbits{512}$\Comment{Generate a random value.}
            \State \Call{Absorb}{$v$}\Comment{Absorb the random value.}
            \State $k \gets$ \Call{Squeeze}{$64$} $\modl$
            \State \textbf{yield} $k$\Comment{Yield a hedged commitment scalar.}
            \End
            \State
            \State \Call{Cyclist}{\textsc{SqueezeKey}($64$), $\epsilon$, $\epsilon$}\Comment{Convert to a keyed duplex.}
            \State
            \State $I \gets [k]G$\Comment{Calculate the commitment point.}
            \State $S_0 \gets $ \Call{Encrypt}{$I$}\Comment{Encrypt the commitment point.}
            \State
            \State $r \gets$ \Call{Squeeze}{$64$} $\modl$\Comment{Squeeze a challenge scalar.}
            \State $s \gets dr + k$\Comment{Calculate the proof scalar.}
            \State $S_1 \gets $ \Call{Encrypt}{$s$}\Comment{Encrypt the proof scalar.}
            \State
            \State \textbf{return} $S=S_0 || S_1$
        \EndFunction
    \end{algorithmic}
    \label{alg:veil.schnorr-sign}
\end{algorithm}

\subsection{Verifying A Signature}\label{subsec:veil.schnorr-verify}

Verifying a signature is described in full in Alg.~\ref{alg:veil.schnorr-verify}.

\begin{algorithm}[!htp]
    \caption{
        Verifying a signature $S$ with a message $M$ and a public key $Q$.
    }
    \begin{algorithmic}
        \Function{Verify}{$Q, M, S=S_0||S_1$}
            \State \Call{Absorb}{\texttt{veil.schnorr}}\Comment{Initialize an unkeyed duplex.}
            \State \Call{Absorb}{$Q$}\Comment{Absorb the signer's public key.}
            \State \Call{Absorb}{$M$}\Comment{Absorb the message.}
            \State
            \State \Call{Cyclist}{\textsc{SqueezeKey}($64$), $\epsilon$, $\epsilon$}\Comment{Convert to a keyed duplex.}
            \State
            \State $I \gets $ \Call{Decrypt}{$S_0$}\Comment{Decrypt the commitment point.}
            \State $r \gets$ \Call{Squeeze}{$64$} $\modl$\Comment{Squeeze a challenge scalar.}
            \State
            \State $s \gets $ \Call{Decrypt}{$S_1$}\Comment{Decrypt the proof scalar.}
            \State $I' \gets [s]G - [r]Q$\Comment{Calculate the counterfactual commitment point.}
            \State
            \State \textbf{return} $I' \checkeq I$\Comment{The signature is valid if both points are equal.}
        \EndFunction
    \end{algorithmic}
    \label{alg:veil.schnorr-verify}
\end{algorithm}

\subsection{Constructive Analysis Of \texttt{veil.schnorr}}\label{subsec:veil.schnorr-analysis}

The Schnorr signature scheme is the application of the Fiat-Shamir transform to the Schnorr identification scheme.

Unlike Construction 13.12 of~\cite[p. 482]{katz2020}, \texttt{veil.schnorr} transmits the commitment point $I$ as part
of the signature and the verifier calculates $I'$ vs transmitting the challenge scalar $r$ and calculating $r'$.
In this way, \texttt{veil.schnorr} is closer to EdDSA~\cite{brendel2021} or the Schnorr variant proposed by Hamburg
in~\cite{hamburg2017}.

\subsection{UF-CMA Security}\label{subsec:veil.schnorr-uf-cma}

Per Theorem 13.10 of~\cite[p. 478]{katz2020}, this construction is UF-CMA secure if the Schnorr identification scheme
is secure and the hash function is secure:

\begin{displayquote}
    Let $\Pi$ be an identification scheme, and let $\Pi'$ be the signature scheme that results by applying the
    Fiat-Shamir transform to it.
    If $\Pi$ is secure and $H$ is modeled as a random oracle, then $\Pi'$ is secure.
\end{displayquote}

Per Theorem 13.11 of~\cite[p. 481]{katz2020}, the security of the Schnorr identification scheme is conditioned on the
hardness of the discrete logarithm problem:

\begin{displayquote}
    If the discrete-logarithm problem is hard relative to $\mathcal{G}$, then the Schnorr identification scheme is
    secure.
\end{displayquote}

Per~\cite[Sec. 5.10]{bertoni2011sponge}, Cyclist is a suitable random oracle if the underlying permutation is
indistinguishable from a random permutation.
Thus, \texttt{veil.schnorr} is UF-CMA if the discrete-logarithm problem is hard relative to P-256 and Keccak-\emph{p} is
indistinguishable from a random permutation.

\subsection{sUF-CMA Security}\label{subsec:veil.schnorr-suf-cma}

Some Schnorr/EdDSA implementations (e.g.\ Ed25519) suffer from malleability issues, allowing for multiple valid
signatures for a given signer and message~\cite{brendel2021}.
Chalkias et al.~\cite{chalkias2020} describe a strict verification function for Ed25519 which achieves sUF-CMA security
in addition to strong binding:

\begin{displayquote}
    \begin{enumerate}
        \item Reject the signature if $S \not\in \{0,\ldots,L-1\}$.
        \item Reject the signature if the public key $A$ is one of 8 small order points.
        \item Reject the signature if $A$ or $R$ are non-canonical.
        \item Compute the hash $\text{SHA2}_{512}(R||A||M)$ and reduce it mod $L$ to get a scalar $h$.
        \item Accept if $8(S \cdot B)-8R-8(h \cdot A)=0$.
    \end{enumerate}
\end{displayquote}

Rejecting $S \geq L$ makes the scheme sUF-CMA secure, and rejecting small order $A$ values makes the scheme strongly
binding.
\texttt{veil.schnorr}'s use of canonical point and scalar encoding routines obviate the need for these checks.
Likewise, P-256 is a prime order group, which obviates the need for cofactoring in verification.

When implemented with a prime order group and canonical encoding routines, the Schnorr signature scheme is strongly
unforgeable under chosen message attack (sUF-CMA) in the random oracle model~\cite{pointcheval2000} and even with
practical cryptographic hash functions~\cite{neven2009}.

\subsection{Key Privacy}\label{subsec:veil.schnorr-key-privacy}

The EdDSA variant (i.e.\ $S=(I,s)$) is used over the traditional Schnorr construction (i.e.\ $S=(r,s)$) to enable the
variable-time computation of $I'=[s]G - [r]Q$, which provides a ~30\% performance improvement.
That construction, however, allows for the recovery of the signing public key given a signature and a message: given the
commitment point $I$, one can calculate $Q=-[r^{-1}](I - [s]G)$.

For Veil, this behavior is not desirable.
A global passive adversary should not be able to discover the identity of a signer from a signed message.

To eliminate this possibility, \texttt{veil.schnorr} encrypts both components of the signature with a duplex keyed with
the signer's public key in addition to the message.
An attack which recovers the plaintext of either signature component in the absence of the public key would imply that
Cyclist is not IND-CPA\@.

\subsection{Indistinguishability From Random Noise}\label{subsec:veil.schnorr-indistinguishability}

Given that both signature components are encrypted with Cyclist, an attack which distinguishes between a
\texttt{veil.schnorr} and random noise would also imply that Cyclist is not IND-CPA\@.

    \section{Encrypted Headers}\label{sec:veil.sres}

\texttt{veil.sres} implements a single-receiver, deniable signcryption scheme which Veil uses to
encrypt message headers. It integrates an ephemeral ECDH KEM, a Cyclist DEM, and a
designated-verifier Schnorr signature scheme to provide multi-user insider security.

\subsection{Encrypting A Header}\label{subsec:veil.sres-encrypt}

Encrypting a header is described in full in Alg.~\ref{alg:veil.sres-encrypt}\@.

\begin{algorithm}
    \caption{Encrypting a header with sender's key pair $(d_S, Q_S)$, ephemeral key pair
        $(d_E, Q_E)$, receiver's public key $Q_R$, nonce $N$, and plaintext $P$.}
    \begin{algorithmic}[1]
        \Function{EncryptHeader}{$(d_S, Q_S), (d_E, Q_E), Q_R, N, P$}
        \State \Call{Absorb}{\texttt{veil.sres}}\Comment{Initialize an unkeyed duplex.}
        \State \Call{Absorb}{$Q_S$}\label{alg:veil.sres-encrypt-bind-sender}\Comment{Absorb the sender's public key.}
        \State \Call{Absorb}{$Q_R$}\Comment{Absorb the receiver's public key.}
        \State \Call{Absorb}{$N$}\Comment{Absorb the nonce.}
        \State \Call{Absorb}{$d_S[Q_R]$}\Comment{Absorb the static ECDH shared secret.}
        \State \Call{Cyclist}{\textsc{SqueezeKey}($64$), $\epsilon$, $\epsilon$}\Comment{Convert to a keyed duplex.}
        \State
        \State $C_0 \gets$ \Call{Encrypt}{$Q_E$}\Comment{Encrypt the ephemeral public key.}
        \State \Call{Absorb}{$d_E[Q_R]$}\Comment{Absorb the ephemeral ECDH shared secret.}
        \State $C_1 \gets$ \Call{Encrypt}{$P$}\Comment{Encrypt the plaintext.}
        \State
        \Clone \Comment{Clone the duplex's state.}
        \State \Call{Absorb}{$d_S$}\Comment{Absorb the sender's private key.}
        \State $v \rgets \allbits{512}$\Comment{Generate a random value.}
        \State \Call{Absorb}{$v$}\Comment{Absorb the random value.}
        \State $k \gets$ \Call{Squeeze}{$32$} $\modl$\Comment{Squeeze a commitment scalar.}
        \State \textbf{yield} $k$
        \End
        \State
        \State $I \gets [k]G$\Comment{Calculate the commitment point.}
        \State $S_0 \gets$ \Call{Encrypt}{$I$}\Comment{Encrypt the commitment point.}
        \State
        \State $r \gets$ \Call{Squeeze}{$32$} $\modl$
        \label{alg:veil.sres-encrypt-challenge}
        \Comment{Squeeze a challenge scalar.}
        \State $s \gets d_S{r}+k$\Comment{Calculate the proof scalar.}
        \State
        \State $X \gets [s]Q_R$\Comment{Calculate the proof point.}
        \State $S_1 \gets$ \Call{Encrypt}{$X$}\Comment{Encrypt the proof point.}
        \State
        \State \textbf{return} $C_0||C_1||S_0||S_1$
        \EndFunction
    \end{algorithmic}
    \label{alg:veil.sres-encrypt}
\end{algorithm}

\subsection{Decrypting A Header}\label{subsec:veil.sres-decrypt}

Decrypting a header is described in full in Alg.~\ref{alg:veil.sres-decrypt}\@.

\begin{algorithm}
    \caption{Decrypting a header with receiver's key pair $(d_R, Q_R)$, sender's public key $Q_S$,
        nonce $N$, and ciphertext $C_0||C_1||S_0||S_1$.}
    \begin{algorithmic}[1]
        \Function{DecryptHeader}{$(d_R, Q_R), Q_S, N, C_0||C_1||S_0||S_1$}
        \State \Call{Absorb}{\texttt{veil.sres}}\Comment{Initialize an unkeyed duplex.}
        \State \Call{Absorb}{$Q_S$}\label{alg:veil.sres-decrypt-bind-sender}\Comment{Absorb the sender's public key.}
        \State \Call{Absorb}{$Q_R$}\Comment{Absorb the receiver's public key.}
        \State \Call{Absorb}{$N$}\Comment{Absorb the nonce.}
        \State
        \State \Call{Absorb}{$d_R[Q_S]$}\Comment{Absorb the static ECDH shared secret.}
        \State \Call{Cyclist}{\textsc{SqueezeKey}($64$), $\epsilon$, $\epsilon$}\Comment{Convert to a keyed duplex.}
        \State
        \State ${Q_E}' \gets$ \Call{Decrypt}{$C_0$}\Comment{Decrypt the ephemeral public key.}
        \State \Call{Absorb}{$d_R[{Q_E}']$}\Comment{Absorb the ephemeral ECDH shared secret.}
        \State $P' \gets$ \Call{Decrypt}{$C_1$}\Comment{Decrypt the plaintext.}
        \State
        \State $I \gets$ \Call{Decrypt}{$S_0$}\Comment{Decrypt the commitment point.}
        \State $r' \gets$ \Call{Squeeze}{$32$} $\modl$\Comment{Squeeze a challenge scalar.}
        \State
        \State $X \gets$ \Call{Decrypt}{$S_1$}\Comment{Decrypt the proof point.}
        \State $X' \gets [d_R](I + [r']Q_S)$\Comment{Re-calculate the proof point.}
        \State
        \If{$X' \checkeq X$}
        \Comment{Ensure the ciphertext is authentic.}
        \State \textbf{return} ${Q_E}', P'$
        \Else
        \State \textbf{return} $\bot$
        \EndIf
        \EndFunction
    \end{algorithmic}
    \label{alg:veil.sres-decrypt}
\end{algorithm}

\subsection{Constructive Analysis Of \texttt{veil.sres}}\label{subsec:veil.sres-analysis}

\texttt{veil.sres} is an integration of two well-known constructions: an ECIES-style hybrid public
key encryption scheme and a designated-verifier Schnorr signature scheme.

The initial portion of \texttt{veil.sres} is equivalent to ECIES (see Construction 12.23 of~\cite[p.
    435]{katz2020}), \@(with the commitment point $I$ as an addition to the ciphertext, and the
challenge scalar $r$ serving as the authentication tag for the data encapsulation mechanism) and is
IND-CCA2 secure (see Corollary 12.14 of~\cite[p. 436]{katz2020}).

The latter portion of \texttt{veil.sres} is a designated-verifier Schnorr signature scheme which
adapts an EdDSA-style Schnorr signature scheme by multiplying the proof scalar $s$ by the receiver's
public key $Q_R$ to produce a designated-verifier point $X$~\cite{steinfeld2004}. The EdDSA-style
Schnorr signature is sUF-CMA secure when implemented in a prime order group and a cryptographic hash
function~\cite{brendel2021, chalkias2020, pointcheval2000, neven2009} (see also
Sec.~\ref{sec:veil.schnorr})\@.

\subsection{Multi-User Confidentiality}\label{subsec:veil.sres-conf}

One of the two main goals of the \texttt{veil.sres} is confidentiality in the multi-user setting
(see Sec.~\ref{subsec:sec-conf}), or the inability of an adversary $\Adversary$ to learn information
about plaintexts.

\subsubsection{Outsider Confidentiality}

First, we evaluate the confidentiality of \texttt{veil.sres} in the multi-user outsider setting (see
Sec.~\ref{subsubsec:sec-conf-outsider}), in which the adversary $\Adversary$ knows the public keys
of all users but none of their private keys~\cite[p. 44]{baek2010}\@.

The classic multi-user attack on the generic Encrypt-Then-Sign ($\EtS$) construction sees
$\Adversary$ strip the signature $\sigma$ from the challenge ciphertext
\[
    C=(c,\sigma,Q_S,Q_R)
\]
and replace it with
\[
    \sigma' \rgets \text{Sign}(d_{\Adversary},c)
\]
to produce an attacker ciphertext
\[
    C'=(c,\sigma',Q_{\Adversary},Q_R)
\]
at which point $\Adversary$ can trick the receiver into decrypting the result and giving
$\Adversary$ to the randomly-chosen plaintext $m_0 \lor m_1$~\cite[p. 40]{an2010}. This attack is
not possible with \texttt{veil.sres}\@, as the sender's public key is strongly bound during
encryption (see Alg.~\ref{alg:veil.sres-encrypt}, Line~\ref{alg:veil.sres-encrypt-bind-sender}) and
decryption (see Alg.~\ref{alg:veil.sres-decrypt}, Line~\ref{alg:veil.sres-decrypt-bind-sender}).

$\Adversary$ is unable to forge valid signatures for existing ciphertexts, limiting them to passive
attacks. A passive attack on any of the three components of \texttt{veil.sres} ciphertexts--$C$,
$S_0$, $S_1$--would only be possible if Cyclist is not IND-CPA secure.

Therefore, \texttt{veil.sres} provides confidentiality in the multi-user outsider setting.

\subsubsection{Insider Confidentiality}

Next, we evaluate the confidentiality of \texttt{veil.sres} in the multi-user insider setting (see
Sec.~\ref{subsubsec:sec-conf-insider}), in which the adversary $\Adversary$ knows the sender's
private key in addition to the public keys of both users~\cite[p. 45--46]{baek2010}\@.

$\Adversary$ cannot decrypt the message by themselves, as they do not know either $d_E$ or $d_R$ and
cannot calculate the ECDH shared secret $[d_E]Q_R=[d_R]Q_E=[d_E{d_R}G]$.

$\Adversary$ also cannot trick the receiver into decrypting an equivalent message by replacing the
signature, despite $\Adversary\text{'s}$ ability to use $d_S$ to create new signatures. In order to
generate a valid signature on a ciphertext $c'$ (e.g.\ $c'=c||1$), $\Adversary$ would have to
squeeze a valid challenge scalar $r'$ from the duplex state (see Alg.~\ref{alg:veil.sres-encrypt},
Line~\ref{alg:veil.sres-encrypt-challenge}). Unlike the signature hash function in the generic
$\EtS$ composition, however, the duplex state is cryptographically dependent on values $\Adversary$
does not know, specifically the ECDH shared secret $[d_E]Q_S$ \@(via the \textsc{Absorb} operation)
and the plaintext $P$ (via the \textsc{Encrypt} operation).

Therefore, \texttt{veil.sres} provides confidentiality in the multi-user insider setting.

\subsection{Multi-User Authenticity}\label{subsec:veil.sres-auth}

The second of the two main goals of the \texttt{veil.sres} is authenticity in the multi-user setting
(see Sec.~\ref{subsec:sec-auth}), or the inability of an adversary $\Adversary$ to forge valid
ciphertexts.

\subsubsection{Outsider Authenticity}

First, we evaluate the authenticity of \texttt{veil.sres} in the multi-user outsider setting (see
Sec.~\ref{subsubsec:sec-auth-outsider}), in which the adversary $\Adversary$ knows the public keys
of all users but none of their private keys~\cite[p. 47]{baek2010}\@.

Because the Schnorr signature scheme is sUF-CMA secure, it is infeasible for $\Adversary$ to forge a
signature for a new message or modify an existing signature for an existing message. Therefore,
\texttt{veil.sres} provides authenticity in the multi-user outsider setting.

\subsubsection{Insider Authenticity}

Next, we evaluate the authenticity of \texttt{veil.sres} in the multi-user insider setting (see
Sec.~\ref{subsubsec:sec-auth-insider}), in which the adversary $\Adversary$ knows the receiver's
private key in addition to the public keys of both users~\cite[p. 48]{baek2010}\@.

Again, the Schnorr signature scheme is sUF-CMA secure and the signature is created using the
signer's private key. The receiver (or $\Adversary$ in possession of the receiver's private key)
cannot forge signatures for new messages. Therefore, \texttt{veil.sres} provides authenticity in the
multi-user insider setting.

\subsection{Limited Deniability}\label{subsec:veil.sres-deniability}

\texttt{veil.sres}{'s} use of a designated-verifier Schnorr scheme provides limited deniability for
senders (see Sec.~\ref{subsec:security-deniability}). Without revealing $d_R$, the receiver cannot
prove the authenticity of a message \@(including the identity of its sender) to a third party.

\subsection{Indistinguishability From Random Noise}\label{subsec:veil.sres-indistinguishability}

All of the components of a \texttt{veil.sres} ciphertext--$C$, $S_0$, and $S_1$--are Cyclist
ciphertexts. An adversary in the outsider setting \@(i.e.\ knowing only public keys) is unable to
calculate any of the key material used to produce the ciphertexts; a distinguishing attack is
infeasible if Cyclist is IND-CPA secure.

\subsection{Re-use Of Ephemeral Keys}\label{subsec:veil.sres-re-using-ephemeral-keys}

The re-use of an ephemeral key pair $(d_E, Q_E)$ across multiple ciphertexts does not impair the
confidentiality of the scheme provided $(N, Q_R)$ pairs are not re-used~\cite{bellare2003}\@. An
adversary who compromises a retained ephemeral private key would be able to decrypt all messages the
sender encrypted using that ephemeral key, thus the forward sender security is bounded by the
sender's retention of the ephemeral private key.

    \section{Encrypted Messages}\label{sec:veil.mres}

\texttt{veil.mres} implements a multi-recipient signcryption scheme.

\subsection{Encrypting A Message}\label{subsec:veil.mres-encrypt}

Encrypting a message is described in full in Alg.~\ref{alg:veil.mres-encrypt}.

\begin{algorithm}[ht]
    \caption{
        Encrypting a message with sender's key pair $(d_S, Q_S)$, receiver public keys $Q_{R^0}..Q_{R^n}$, padding
        length $N_P$, and plaintext $P$.
    }
    \begin{algorithmic}
        \Function{EncryptMessage}{$(d_S, Q_S), Q_{R^0}..Q_{R^n}, N_P, P$}
            \State \Call{Cyclist}{\texttt{veil.mres}, $\epsilon$, $\epsilon$}\Comment{Initialize duplex with constant key.}
            \State \Call{Absorb}{$Q_S$}\label{alg:veil.mres-encrypt-bind-sender}\Comment{Absorb the sender's public key.}
            \State
            \Clone \Comment{Clone the duplex's state.}
            \State \Call{Absorb}{$d_S$}\Comment{Absorb the sender's private key.}
            \State $v \rgets \allbits{512}$\Comment{Generate a random value.}
            \State \Call{Absorb}{$v$}\Comment{Absorb the random value.}
            \State $k \gets$ \Call{SqueezeKey}{$64$} $\modl$\Comment{Squeeze a commitment scalar.}
            \State $d_E \gets$ \Call{SqueezeKey}{$64$} $\modl$\Comment{Squeeze an ephemeral private key.}
            \State $K \gets$ \Call{Squeeze}{$32$}\Comment{Squeeze a data encryption key.}
            \State \textbf{yield} $(k, d_E, K)$
            \End
            \State
            \State $Q_E = [d_E]G$\Comment{Mask and absorb the ephemeral public key.}
            \State $C \rgets $\Call{Mask}{$Q_E$}
            \State \Call{Absorb}{$C$}
            \State
            \State $H = K||N_Q||N_P$\Comment{Encode the DEK and params in a header.}
            \ForAll{$Q_{R^i} \in \{Q_{R^0}..Q_{R^n}\}$}
                \Comment{Encrypt the header for each receiver.}
                \State $D \gets$\Call{Squeeze}{$16$}
                \State $H \gets $\Call{EncryptHeader}{$(d_S, Q_S), (d_E, Q_E), Q_{R^i}, H, D$}
                \State \Call{Absorb}{$H$}
                \State $C \gets C||H$
            \EndFor
            \State
            \State $y \rgets \allbits{N_P}$\Comment{Absorb and append random padding.}
            \State \Call{AbsorbMore}{$y, 16$}
            \State $C \gets C||y$
            \State
            \State \Call{Cyclist}{$K$, $\epsilon$, $\epsilon$}\Comment{Re-key the duplex with the DEK.}
            \ForAll{32-KiB blocks $p \in P$}
                \State $C \gets C||$\Call{Encrypt}{$p$}\Comment{Encrypt and tag each block.}
                \State $C \gets C||$\Call{Squeeze}{$16$}
                \State \Call{Ratchet}{}
            \EndFor
            \State
            \State $I \gets [k]G$\Comment{Calculate and encrypt the commitment point.}
            \State $C \gets C||$\Call{Encrypt}{$I$}
            \State
            \State $r \gets$ \Call{SqueezeKey}{$64$} $\modl$\label{alg:veil.mres-encrypt-challenge}\Comment{Squeeze a challenge scalar.}
            \State $s \gets {d_S}r + k$\Comment{Calculate and encrypt the proof scalar.}
            \State $C \gets C||$\Call{Encrypt}{$s$}
            \State \textbf{return} $C$
        \EndFunction
    \end{algorithmic}
    \label{alg:veil.mres-encrypt}
\end{algorithm}

\subsection{Decrypting A Message}\label{subsec:veil.mres-decrypt}

Decrypting a signature is described in full in Alg.~\ref{alg:veil.mres-decrypt}.

\begin{algorithm}[ht]
    \caption{
        Decrypting a message with receiver's key pair $(d_R, Q_R)$, sender's public key $Q_S$, and ciphertext $C$.
    }
    \begin{algorithmic}
        \Function{DecryptMessage}{$(d_R, Q_R), Q_S, C$}
            \State \Call{Cyclist}{\texttt{veil.mres}, $\epsilon$, $\epsilon$}\Comment{Initialize duplex with constant key.}
            \State \Call{Absorb}{$Q_S$}\label{alg:veil.mres-decrypt-bind-sender}\Comment{Absorb the sender's public key.}
            \State \Call{Absorb}{$C[0..31]$}
            \State $Q_E \gets $\Call{Unmask}{$C[0..31]$}\Comment{Unmask and decode the ephemeral public key.}
            \State
            \ForAll{possible encrypted headers $h \in C$}
                \State $D \gets$\Call{Squeeze}{$16$}
                \State $x \gets $ \Call{DecryptHeader}{$(d_R, Q_R), Q_S, Q_E, h, D$}
                \If{$x \not= \bot$}
                    \State $K || N_Q || N_P \gets x$
                \EndIf
                \State \Call{Absorb}{$h$}
            \EndFor
            \State
            \State \Call{AbsorbMore}{$C_[32+N_Q..32+N_Q+N_P], 16$}\Comment{Absorb padding.}
            \State $C \gets C_{[32+N_Q+N_P]..}$\Comment{Skip to the message beginning.}
            \State
            \State \Call{Cyclist}{$K$, $\epsilon$, $\epsilon$}\Comment{Re-key the duplex with the DEK.}
            \State
            \State $P' = \epsilon$\Comment{Initialize a buffer for the plaintext.}
            \ForAll{32KiB+16 blocks $c||t \in C$}
                \State $P' \gets P'||$\Call{Decrypt}{$c$}
                \If{\Call{Squeeze}{$16$} $ \not= t$}
                    \Comment{Check each block's tag.}
                    \State \textbf{return} $\bot$
                \EndIf
                \State \Call{Ratchet}{}
            \EndFor
            \State
            \State $S_0||S_1 \gets C$
            \State $I \gets $ \Call{Decrypt}{$S_0$}\Comment{Decrypt the commitment point.}
            \State $r \gets$ \Call{SqueezeKey}{$64$} $\modl$\Comment{Squeeze a challenge scalar.}
            \State
            \State $s \gets $ \Call{Decrypt}{$S_1$}\Comment{Decrypt the proof scalar.}
            \State $I' \gets [s]G - [r]Q_E$\Comment{Calculate the counterfactual commitment point.}
            \State
            \If{$I' \checkeq I$}
                \Comment{Verify the signature.}
                \State \textbf{return} $P'$ \Comment{Return the plaintext.}
            \Else
                \State \textbf{return} $\bot$
            \EndIf
        \EndFunction
    \end{algorithmic}
    \label{alg:veil.mres-decrypt}
\end{algorithm}

\subsection{Constructive Analysis Of \texttt{veil.mres}}\label{subsec:veil.mres-analysis}

\texttt{veil.mres} is an integration of two well-known constructions: a multi-recipient hybrid encryption scheme and an
EdDSA-style Schnorr signature scheme.

The initial portion of \texttt{veil.mres} is a multi-recipient hybrid encryption scheme, with per-receiver copies of a
symmetric data encryption key (DEK) encrypted in headers with the receivers' public
keys~\cite{kurosawa2002, bellare2003, bellare2007, rfc4880}.
The headers are encrypted with the \texttt{veil.sres} construction (see~\ref{sec:veil.sres}), which provides full
insider security (i.e.\ IND-CCA2 and sUF-CMA in the multi-user insider setting).
The message itself is divided into a sequence of 32KiB blocks, each encrypted as a sequence of Xoodyak
\textsc{Encrypt}/\textsc{Squeeze}/\textsc{Ratchet} operations and each with a 16-byte authentication tag, which is
IND-CCA2 secure.

The latter portion of \texttt{veil.mres} is an EdDSA-style Schnorr signature scheme.
The EdDSA-style Schnorr signature is sUF-CMA secure when implemented in a prime order group and a cryptographic hash
function~\cite{brendel2021, chalkias2020, pointcheval2000, neven2009} (see also~\ref{sec:veil.schnorr}).

\subsection{Multi-User Confidentiality}\label{subsec:veil.mres-conf}

One of the two main goals of the \texttt{veil.mres} is confidentiality in the multi-user setting
(see~\ref{subsec:sec-conf}), or the inability of an adversary $\Adversary$ to learn information about plaintexts.
As \texttt{veil.mres} is a multi-recipient scheme, we adopt Bellare et al.'s adaptations of the multi-user
setting, in which $\Adversary$ may compromise a subset of receivers~\cite{bellare2007}.

\subsubsection{Outsider Confidentiality}

First, we evaluate the confidentiality of \texttt{veil.mres} in the multi-user outsider setting
(see~\ref{subsubsec:sec-conf-outsider}), in which the adversary $\Adversary$ knows the public keys of all users but none
of their private keys~\cite[p. 44]{baek2010}.

As with \texttt{veil.sres} (see~\ref{subsec:veil.sres-conf}), \texttt{veil.mres} superficially resembles an
Encrypt-Then-Sign ($\EtS$) scheme, which are vulnerable to an attack where by $\Adversary$ strips the signature from the
challenge ciphertext and either signs it themselves or tricks the sender into signing it, thereby creating a new
ciphertext they can then trick the receiver into decrypting for them.
Again, as with \texttt{veil.sres}, the identity of the sender is strongly bound during encryption
encryption (see Alg.~\ref{alg:veil.mres-encrypt}, Line~\ref{alg:veil.mres-encrypt-bind-sender}) and decryption
(see Alg.~\ref{alg:veil.mres-decrypt}, Line~\ref{alg:veil.mres-decrypt-bind-sender}), making this infeasible.

$\Adversary$ is unable to forge valid signatures for existing ciphertexts, limiting them to passive attacks.
\texttt{veil.mres} ciphertexts consist of ephemeral keys, encrypted headers, random padding, encrypted message blocks,
and encrypted signature points.
Each component of the ciphertext is dependent on the previous inputs (including the headers, which use
\textsc{Squeeze}-derived authenticated data to link the \texttt{veil.sres} ciphertexts to the \texttt{veil.mres} state).
A passive attack on any of those would only be possible if Xoodyak is not IND-CPA secure.

\subsubsection{Insider Confidentiality}

Next, we evaluate the confidentiality of \texttt{veil.mres} in the multi-user insider setting
(see~\ref{subsubsec:sec-conf-insider}), in which the adversary $\Adversary$ knows the sender's private key in addition
to the public keys of all users~\cite[p. 45--46]{baek2010}.

$\Adversary$ cannot decrypt the message by themselves, as they do not know either $d_E$ or any $d_R$ and cannot decrypt
any of the \texttt{veil.sres}-encrypted headers.

As with \texttt{veil.sres} (see~\ref{subsec:veil.sres-conf}),
$\Adversary$ cannot trick the receiver into decrypting an equivalent message by replacing the signature, despite
$\Adversary$'s ability to use $d_S$ to create new headers.
In order to generate a valid signature on a ciphertext $c'$ (e.g.\ $c'=c||1$), $\Adversary$ would have to squeeze a
valid challenge scalar $r'$ from the duplex state (see Alg.~\ref{alg:veil.mres-encrypt},
line~\ref{alg:veil.mres-encrypt-challenge}).
Unlike the signature hash function in the generic $\EtS$ composition, however, the duplex state is cryptographically
dependent on a value $\Adversary$ does not know, specifically the data encryption key $K$ (via the \textsc{Cyclist}
operation) and the plaintext blocks $p_{0..n}$ (via the \textsc{Encrypt} operation).

Therefore, \texttt{veil.mres} provides confidentiality in the multi-user insider setting.

\subsection{Multi-User Authenticity}\label{subsec:veil.mres-auth}

The second of the two main goals of the \texttt{veil.mres} is authenticity in the multi-user setting
(see~\ref{subsec:sec-auth}), or the inability of an adversary $\Adversary$ to forge valid ciphertexts.

\subsubsection{Outsider Authenticity}

First, we evaluate the authenticity of \texttt{veil.mres} in the multi-user outsider setting
(see~\ref{subsubsec:sec-auth-outsider}), in which the adversary $\Adversary$ knows the public keys of all users but none
of their private keys~\cite[p. 47]{baek2010}.

Because the Schnorr signature scheme is sUF-CMA secure, it is infeasible for $\Adversary$ to forge a signature for a new
message or modify an existing signature for an existing message.
Therefore, \texttt{veil.mres} provides authenticity in the multi-user outsider setting.

\subsubsection{Insider Authenticity}

Next, we evaluate the authenticity of \texttt{veil.mres} in the multi-user insider setting
(see~\ref{subsubsec:sec-auth-insider}), in which the adversary $\Adversary$ knows some receivers' private keys in
addition to the public keys of both users~\cite[p. 48]{baek2010}.

Again, the Schnorr signature scheme is sUF-CMA secure and the signature is created using the ephemeral private key,
which $\Adversary$ does not possess.
The receiver (or $\Adversary$ in possession of the receiver's private key) cannot forge signatures for new messages.
Therefore, \texttt{veil.mres} provides authenticity in the multi-user insider setting.

\subsection{Limited Deniability}\label{subsec:veil.mres-deniability}

The only portion of \texttt{veil.mres} ciphertexts which are creating using the sender's private key (and thus tying
a particular message to their identity) are the \texttt{veil.sres}-encrypted headers.
All other components are creating using the data encryption key or ephemeral private key, neither of which are bound
to identity.
\texttt{veil.sres} provides limited deniability (see~\ref{subsec:veil.sres-deniability}), therefore \texttt{veil.mres}
does as well.

\subsection{Indistinguishability From Random Noise}\label{subsec:veil.mres-indistinguishability}

\texttt{veil.mres} ciphertexts are, unfortunately, not fully indistinguishable from random noise.
The \textsc{Mask} function randomly flips the most and least significant bits of the encoded point, which removes the
most trivial distinguishing attack for the ephemeral public key but does not fully resolve the issue.

\subsubsection{Future Work}

This aspect of the \texttt{veil.mres} would benefit greatly from the development of a bijective Elligator2-style map
for Ristretto points.
Adapting the \textsc{Mask} function to produce an ephemeral key pair with a suitable uniform representative would not
affect its security properties and would eliminate the available distinguishing attack.

\subsection{Partial Decryption}\label{subsec:veil.mres-partial-decryption}

The division of the plaintext stream into blocks takes its inspiration from the CHAIN construction~\cite{hoang2015}, but
the use of Xoodyak allows for a significant reduction in complexity.
Instead of using the nonce and associated data to create a feed-forward ciphertext dependency, the Xoodyak duplex
ensures all encryption operations are cryptographically dependent on the ciphertext of all previous encryption
operations.
Likewise, because the \texttt{veil.mres} ciphertext is terminated with a Schnorr signature (see
Sec.~\ref{sec:veil.schnorr}), using a special operation for the final message block isn't required.

The major limitation of such a system is the possibility of the partial decryption of invalid ciphertexts.
If an attacker flips a bit on the fourth block of a ciphertext, \texttt{veil.mres} will successfully decrypt the first
three before returning an error.
If the end-user interface displays that, the attacker may be successful in radically altering the semantics of an
encrypted message without the user's awareness.
The first three blocks of a message, for example, could say \texttt{PAY MALLORY \$100}, \texttt{GIVE HER YOUR CAR},
\texttt{DO WHAT SHE SAYS}, while the last block might read \texttt{JUST KIDDING}.

    \section{Passphrase-Based Encryption}\label{sec:veil.pbenc}

\texttt{veil.pbenc} implements a memory-hard authenticated encryption scheme to encrypt private keys at rest.

\subsection{Initialization}\label{subsec:veil.pbenc-init}

Initializing a duplex from a passphrase is described in full in Alg.~\ref{alg:veil.pbenc-hash} and
Alg.~\ref{alg:veil.pbenc-init}\@.

\begin{algorithm}
    \caption{
        Producing a hashed block given a counter $C$, a sequence of input blocks $B_0..B_n$, and an output length $N$.
    }
    \begin{algorithmic}
        \Function{Hash}{$C, B_0..B_n, N$}
            \State \Call{Absorb}{\texttt{veil.pbenc.iter}}\Comment{Initialize an unkeyed duplex.}
            \State \Call{Absorb}{$c$}\Comment{Absorb the counter.}
            \State $C \gets C + 1$\Comment{Increment the counter.}
            \ForAll{$B_i \in \{B_0..B_n\}$}
                \State \Call{Absorb}{$B_i$}\Comment{Absorb each piece of data.}
            \EndFor
            \State \textbf{return} \Call{Squeeze}{$N$}\Comment{Squeeze $N$ bytes of output.}
        \EndFunction
    \end{algorithmic}
    \label{alg:veil.pbenc-hash}
\end{algorithm}

\begin{algorithm}
    \caption{
        Initializing a duplex given a passphrase $P$, salt $S$, time parameter $N_T$, space parameter $N_S$, delta
        constant $D = 3$, and block size constant $N_B = 32$.
    }
    \begin{algorithmic}
        \Procedure{InitFromPassphrase}{$P, S, N_T, N_S$}
            \State $C \gets 0$\Comment{Initialize a counter.}
            \State $B \gets [[\texttt{0x00} \times N_B] \times N]$\Comment{Initialize an array of $N$ blocks.}
            \Statex
            \State $B[0] \gets $\Call{Hash}{$C, P, S, N_B$}\Comment{Expand input into buffer.}
            \For{$m \in 1..N_S$}
                \State $B[m] \gets $\Call{Hash}{$C, B[m-1], N_B$}\Comment{Fill remainder of buffer with hash chain.}
            \EndFor
            \Statex
            \For{$t \in 0..N_T$}\Comment{Mix buffer contents.}
                \For{$m \in 0..N_S$}
                    \State $m_{prev} \gets m-1 \bmod N_S$
                    \State $B[m] \gets $\Call{Hash}{$C, B[m_{prev}], B[m], N_B$}\Comment{Hash previous and current blocks.}
                    \For{$i \in 0..D$}
                        \State $r \gets $\Call{Hash}{$C, S, t, m, i, 8$}$ \bmod N_S$\Comment{Hash loop indexes.}
                        \State $B[m] \gets $\Call{Hash}{$C, B[m], B[r], N_B$}\Comment{Hash random and current blocks.}
                    \EndFor
                \EndFor
            \EndFor
            \Statex
            \State \Call{Absorb}{\texttt{veil.pbenc}}\Comment{Initialize an unkeyed duplex.}
            \State \Call{Absorb}{$B[N_S-1]$}\Comment{Extract output from buffer.}
            \State \Call{Cyclist}{\textsc{SqueezeKey}($64$), $\epsilon$, $\epsilon$}\Comment{Convert to a keyed duplex.}
        \EndProcedure
    \end{algorithmic}
    \label{alg:veil.pbenc-init}
\end{algorithm}

\subsection{Encrypting A Private Key}\label{subsec:veil.pbenc-encrypt}

Encrypting a private key is described in full in Alg.~\ref{alg:veil.pbenc-encrypt}\@.

\begin{algorithm}
    \caption{
        Encrypting a private key given a passphrase $P$, time parameter $N_T$, space parameter $N_S$, and
        private key $d$.
    }
    \begin{algorithmic}
        \Function{EncryptPrivateKey}{$P, N_T, N_S, d$}
            \State $S \rgets \allbits{128}$\Comment{Generate a random salt.}
            \State \Call{InitFromPassphrase}{$P, S, N_T, N_S$}\Comment{Initialize the duplex.}
            \State $C \gets$\Call{Encrypt}{$d$}\Comment{Encrypt the private key.}
            \State $T \gets$\Call{Squeeze}{$16$}\Comment{Squeeze an authentication tag.}
            \State \textbf{return} $N_T || N_S || S || C ||T$
        \EndFunction
    \end{algorithmic}
    \label{alg:veil.pbenc-encrypt}
\end{algorithm}

\subsection{Decrypting A Private key}\label{subsec:veil.pbenc-decrypt}

Decrypting a private key is described in full in Alg.~\ref{alg:veil.pbenc-decrypt}\@.

\begin{algorithm}
    \caption{
        Decrypt a private key given a passphrase $P$ and ciphertext $C$.
    }
    \begin{algorithmic}
        \Function{DecryptPrivateKey}{$P, C=N_T || N_S || S || C ||T$}
            \State \Call{InitFromPassphrase}{$P, S, N_T, N_S$}\Comment{Initialize the duplex.}
            \State $d' \gets$\Call{Decrypt}{$C$}\Comment{Decrypt the ciphertext.}
            \State $T' \gets$\Call{Squeeze}{$16$}\Comment{Squeeze a counterfactual tag.}
            \If{$T' \checkeq T$}
                \Comment{Authenticate the ciphertext.}
                \State \textbf{return} $d'$
            \Else
                \State \textbf{return} $\bot$
            \EndIf
        \EndFunction
    \end{algorithmic}
    \label{alg:veil.pbenc-decrypt}
\end{algorithm}

\subsection{Constructive Analysis Of \texttt{veil.pbenc}}\label{subsec:veil.pbenc-analysis}

\texttt{veil.pbenc} is an integration of a memory-hard key derivation function \@(adapted for the cryptographic duplex)
and a standard Cyclist authenticated encryption scheme.

The \textsc{InitFromPassphrase} procedure of \texttt{veil.pbenc} implements balloon hashing, a memory-hard KDF intended
for hashing low-entropy passphrases~\cite{boneh2016}.
Memory-hard functions are a new and active area of cryptographic research, making the evaluation of schemes difficult.
Balloon hashing was selected for its resilience to timing attacks, its reliance on a single hash primitive, and its
relatively well-developed security proofs.

The \textsc{EncryptPrivateKey} and \textsc{DecryptPrivateKey} functions use \textsc{InitFromPassphrase} to initialize
the duplex state, after which they implement a standard Cyclist authenticated encryption scheme, which is IND-CCA2
secure.

    \section{Message Digests \& Authentication Codes}\label{sec:veil.digest}

Veil can create message digests given a set of metadata and a message.

\subsection{Creating A Message Digest}\label{subsec:veil.digest-digest}

Creating a message digest is described in full in Alg.~\ref{alg:veil.digest}.

\begin{algorithm}[h]
    \caption{Creating a message digest with metadata strings $V$ and message $M$.}
    \begin{algorithmic}
        \Function{Digest}{$V,M$}
            \State \Call{Cyclist}{\texttt{veil.digest}, $\epsilon$, $\epsilon$} \Comment{Initialize duplex with constant key.}
            \ForAll{$V_i \in V$}
                \State \Call{Absorb}{$V_i$} \Comment{Absorb metadata strings in order.}
            \EndFor
            \ForAll{$M_i \in M$}
                \State \Call{AbsorbMore}{$M_i$, $16$} \Comment{Absorb message in 16-byte blocks.}
            \EndFor
            \State $D \gets$ \Call{Squeeze}{$64$} \Comment{Squeeze a 64-byte digest.}
            \State \textbf{return} $D$
        \EndFunction
    \end{algorithmic}
    \label{alg:veil.digest}
\end{algorithm}

\subsection{Creating A Message Authentication Code}\label{subsec:veil.digest-mac}

By passing a symmetric key as a metadata string, \texttt{veil.digest} can be adapted to produce message authentication
codes (see Alg.~\ref{alg:veil.digest-mac}).

\begin{algorithm}[h]
    \caption{Creating a message authentication code with key $K$, metadata strings $V$, and message $M$.}
    \begin{algorithmic}
        \Function{Mac}{$K,V,M$}
            \State $T \gets$ \Call{Digest}{$K || V$, $M$} \Comment{Include the key as metadata.}
            \State \textbf{return} $T$
        \EndFunction
    \end{algorithmic}
    \label{alg:veil.digest-mac}
\end{algorithm}

\subsection{Preimage Security and Collision Resistance}\label{subsec:veil.digest-preimage}

The duplex is indistinguishable from a random oracle given that the underlying permutation is indistinguishable from a
random permutation~\cite[Sec. 8.5]{bertoni2011sponge}.
If Xoodoo is strong, an attacker has negligible advantage in finding first or second preimages.

\subsection{sUF-CMA Security}\label{subsec:veil.digest-suf-cma}

Similarly, the security of a duplex as a MAC reduces to the strength of the
permutation~\cite[Sec. 8.6]{bertoni2011sponge}.
An attack which forges a message for a given key and MAC would imply Xoodyak's \textsc{Squeeze} operation is not sUF-CMA
secure.



    \section{References}\label{sec:references}

    \printbibliography
\end{document}
